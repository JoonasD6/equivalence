%Korttien lisääminen komennolla \pair{x}{y}, missä x ja y ovat saman lausekkeen kaksi eri muotoa.
%\pairg{}{}, kun _jälkimmäiseen_ laitetaan kuvaaja %make even simpler to ease addition of cards? Input from text file? separated by ampersand instead of in curly braces?

%switch sille, että minkä luokan kortit printataan.


%\begin{cards}{A}
%\pair{11}{\text{yksitoista}}
%\pair{110}{\text{satakymmenen}}
%\pair{101}{\text{satayksi}}
%\pair{111}{\text{satayksitoista}}
%\pair{1\,000}{\text{tuhat}}
%\pair{1\,100}{\text{tuhat sata}}
%\pair{1\,010}{\text{tuhat kymmenen}}
%\pair{1\,110}{\text{tuhat satakymmenen}}
%\pair{1\,001}{\text{tuhat yksi}}
%\pair{1\,011}{\text{tuhat yksitoista}}
%\pair{1\,101}{\text{tuhat satayksi}}
%\pair{1\,111}{\text{tuhat satayksitoista}}
%\pair{11\,111}{\text{yksitoistatuhatta satayksitoista}}
%\pair{11\,100}{\text{yksitoistatuhatta sata}}
%\pair{11\,010}{\text{yksitoistatuhatta kymmenen}}
%\pair{11\,001}{\text{yksitoistatuhatta yksi}}
%\pair{10\,111}{\text{kymmenentuhatta satayksitoista}}
%\pair{10\,010}{\text{kymmenentuhatta kymmenen}}
%\end{cards}

%\begin{cards}{A} %sum, subtraction, additive inverse, product with natural numbers
%
%\pair{-2-(-5)}{3}
%\pair{5-(-2)}{7}
%\pair{5+(-x)}{5-x}
%\pair{2+(+x)}{2+x}
%\pair{x-y}{-y+x}
%\pair{x-y-x}{-y}
%\pair{x+x-x+x}{2x}
%\pair{x}{x(2-1)}
%\pair{(x+y)-x}{x+(y-x)}
%\pair{-x+(x+y)}{y}
%\pair{-[(-x)\cdot y]}{yx}
%\pair{y\cdot(-x)}{-(xy)}
%\pair{(-2)[(-x)\cdot y]}{2xy}
%\pair{[(-2)(-1)]\cdot 2}{4}
%\pair{-[(-z)\cdot (-y)]}{-(yz)}
%\pair{-(-x)\cdot [-(-y)]}{xy}
%\pair{(-x)(-y)(-z)}{-xyz}
%\pair{(-2x)(-2y)(-2z)}{-8xyz}
%\pair{(-2)[x(-y)]}{2xy}
%\pair{(-2)[x(-y)]}{[x(-2)](-y)}
%\pair{-(2z)\cdot(-(xy))}{2zyx}
%\pair{(x+y)-x+(-y)}{0}
%\pair{(-x)-(-x)}{0}
%\pair{2-2x-(2x)}{2-4x}
%\pair{2-x-2}{-x}
%\pair{x-2-x}{-2}
%\pair{xy-2xy-yx+(-yx)}{-3xy}
%
%\end{cards} %54

%\begin{cards}{B} %cancellation
%	
%\pair{\dfrac{15}{9}}{\dfrac{5}{3}}
%\pair{\dfrac{30}{4}}{\dfrac{15}{2}}
%\pair{\dfrac{2\cdot5\cdot 7}{10\cdot 7 \cdot 8}}{\dfrac{1}{8}}
%\pair{\dfrac{1}{2}\cdot \dfrac{2}{3}\cdot \dfrac{3}{4}}{\dfrac{1}{4}}
%\pair{\dfrac{-570}{9}}{\dfrac{190}{-3}}
%\pair{\dfrac{x}{y}:\dfrac{x^2}{y^2}:\dfrac{x^3}{y^3}}{\dfrac{y^4}{x^4}}
%
%\end{cards}

%\begin{cards}{C}
%\pairg{y=x^2}{x**2}
%\pairg{y=x^2}{1/x}
%\end{cards}



%\begin{kortit}{H} %multiplication notation
%\pair{9\cdot x}{9x}
%\end{kortit}

%\begin{cards}{B} %distributive law of whole numbers  (no powers,´or rational expressions)
%
%\pair{x(a-1)}{ax-x}
%\pair{a(x-1)}{ax-a}
%\pair{a(1-x)}{a-ax} 
%\pair{x(1-a)}{x-ax}
%\pair{3-3\cdot3}{-6}
%
%\pair{3x+2x}{x+4x}
%\pair{x-2(2x+4)}{-3x-8}
%\pair{x+x+x}{3\cdot x}
%\pair{2(2x-4)-x}{3x-8}
%\pair{2t-3t(x-t)\cdot 0}{2t}
%\pair{2(2x-4)-x}{3x-8}
%\pair{-(1-x)-(1-x)}{-2(1-x)}
%\pair{-(2-y)}{y-2}
%\pair{-2(x+1)}{-2x-2}
%\pair{-2(x-1)}{-2x+2}
%\pair{-(-2+y)}{2-y}
%\pair{2t\cdot 0-3t(x-1)}{-3tx+3t}
%\pair{t(1-2)x}{-tx}
%\pair{x+a(1+x)}{a+x(a+1)}
%\pair{2y(x-1)}{2xy-2y}
%\pair{2x(y-1)}{2xy-2x}
%\pair{(x+y)(x+y)}{xx+2xy+yy}
%\pair{(x-y)(x-y)}{xx-2xy+yy}
%\pair{(x+y)(x-y)}{xx-yy}
%\pair{x(y+1)-y(x+1)}{x-y}
%\pair{-(x-1)-(x-1)}{-2x+2}
%
%\pair{2(2x-4)-x}{3x-8}
%\end{cards} %54

%\pair{2(6-1)}{10}
%\pair{3\cdot3-3}{6}
%\pair{(3-3)\cdot3}{0}

\begin{cards}{C}
\pair{y(y^2-1)}{y^3-y}
\pair{y^2(y-1)}{y^3-y^2}
\pair{y(y^2-1)}{(y^2+y)(y-1)}
\pair{(y+1)^2}{y^2+1+2y}
\pair{(y-1)^2}{y^2-2y+1}
\pair{(y+1)^3}{y^3+3y^2+3y+1}
\pair{(y+1)(y-1)}{y^2-1}
\pair{y(1-y)}{-y^2+y}
\pair{y^2-y}{y(y-1)}
\pair{(1-y)^2}{y^2+1-2y}
\pair{y-y(1-y)}{y^2}
\pair{y^2-y^2(1-y^2)}{y^4}
\pair{y^3\cdot y^2}{y^5}
\pair{(-1-y)y}{-y-y^2}
\pair{y-y^2}{(1-y)y}
\pair{y-y(y+1)}{-y^2}
\pair{y^2-y^2(y^2+1)}{-y^4}
\pair{y^2-4}{(y+2)(y-2)}
\end{cards}

%
%\begin{cards}{E} %lisää määrittelyjoukot?
%
%
%\pair{\dfrac{1+x}{x}}{1+1/x}
%\pair{1/(1+1/x))}{x/(x+1)}
%
%\pair{\dfrac{3x^2}{x}}{3x}
%\pair{\dfrac{3x^2}{6x}}{\dfrac{x}{2}}
%\pair{\dfrac{3x}{x^2}}{\dfrac{3}{x}}
%\pair{\dfrac{6x}{3x^2}}{\dfrac{2}{x}}
%
%\pair{\dfrac{6x^2}{x}}{6x}
%
%\pair{\dfrac{x-1}{x^2-1}}{\dfrac{1}{x+1}}
%\pair{\dfrac{x-1}{x^2-x}}{\dfrac{1}{x}}
%\pair{\dfrac{x-1}{x^2-x^2}}{\text{ei määritelty}}
%\pair{\dfrac{\dfrac{x^3}{10}}{\dfrac{x}{5}}}{\dfrac{x^2}{2}}
%\pair{\dfrac{\dfrac{1+x}{x}}{\dfrac{x^2+x}{x^2}}}{1}
%
%\pair{\dfrac{x^2(x-1)}{x+1}\cdot \dfrac{x+1}{x^2}}{x-1}
%\pair{(x-a)(x-b)(x-c)\cdot \ldots \cdot (x-å)}{0}
%
%\pair{\dfrac{1}{x}+\dfrac{1}{x^2}}{\dfrac{x+1}{x^2}}
%\pair{\dfrac{1}{x}-\dfrac{1}{x^2}}{\dfrac{\dfrac{-2}{x^2}-\dfrac{-2}{x}}{2}}
%\pair{\dfrac{1}{x}\cdot \dfrac{1}{x^2}}{\dfrac{1}{x^3}}
%\pair{\dfrac{1}{x}:\dfrac{1}{x^2}}{x}
%
%
%\pair{\dfrac{x^2-x+x(x+1)}{x^2}}{\dfrac{x-1+(x+1)}{x}}
%
%\pair{\dfrac{x-1}{x}+\dfrac{x+1}{x}}{2}
%\pair{\dfrac{x-1}{x}-\dfrac{x+1}{x}}{-\dfrac{2}{x}}
%\pair{\dfrac{x-1}{x}\cdot \dfrac{x+1}{x}}{1-\dfrac{1}{x^2}}
%\pair{\dfrac{x-1}{x}:\dfrac{x+1}{x}}{\dfrac{x-1}{x+1}}
%
%\pair{\dfrac{x}{x-1}+\dfrac{x-1}{x}}{\dfrac{2x^2-2x+1}{x^2-x}}
%\pair{\dfrac{x}{x-1}-\dfrac{x-1}{x}}{\dfrac{2x-1}{x^2-x}}
%\pair{\dfrac{x}{x-1}\cdot \dfrac{x-1}{x}}{1}
%\pair{\dfrac{x}{x-1}:\dfrac{x-1}{x}}{\dfrac{x^2}{x^2-2x+1}}
%
%\pair{(a-1/a):(1-1/a)}{a+1}
%\end{cards}%54
%\pair{\dfrac{3x}{x}}{3}
%\pair{\dfrac{1}{3}x}{\dfrac{x}{3}}
%\pair{\dfrac{\dfrac{5}{6}}{\dfrac{6}{5}}}{\dfrac{25}{36}}




%\begin{cards}{M} %pari, muutama aksioomaa ja laskusääntöä, ei käänteislukua tai osittelulakia, kertolasku luonnollisilla
%
%
%\end{cards}




%\pair{\dfrac{x-1}{1-x}}{-1} %B:tä jo?


%paljon murtolukuja eri muodoissa
%
%kertomerkki -yms. merkitsemiskäytännöt, selvä ero pystyyn kirjoitetutlle ja kursiiville
% yksikköpyörittelyä
%
%

%

%
%
%\pair{(-1)(-2)}{2}
%
%\pair{(-1)(-2)(-1)}{-2}

%
%
%\kB{\dfrac{5}{10}}{\dfrac{6}{10}}
%
%\kB{\dfrac{2x}{x}}{\dfrac{-4x}{2x}}
%
%\setcounter{lauseke}{1}
%

%\pair{1\frac{1}{2}}{1+\frac{1}{1}} %+murtoluvusta desimaaliiin jne.

%mikä aksiooma?
%\begin{kortit}{E}
%%\pair{2+3}{3+2}
%%\pair{x+y}{y+x}
%\pair{x^2+3}{3+x^2}
%\pair{(-1)+3}{3+(-1)}
%\pair{(1+2)+3}{3+(1+2)}
%\pair{(a+b)+c}{(b+a)+c}
%\pair{(a+b)+c}{a+(b+c)}
%\pair{(a+b)+c}{c+(a+b)}
%\pair{-a+b}{b+(-a)}
%\pair{\dfrac{1+x}{x}}{\dfrac{x+1}{x}}
%\pair{-\pi+x^5}{x^5+(-\pi)}
%\pair{2\sin(x)+\sqrt{y}}{\sqrt{y}+2\sin(x)}
%
%\pair{b+(-b)}{b-b}
%
%\pair{b+(-b)}{0}
%
%\pair{x+(-y)}{x-y}
%
%\pair{(y+(-7))+x}{y+(-7)+x}
%\pair{(y+(-7))+x}{y+((-7)+x)}
%\pair{(y+(-7))+x}{((-7)+y)+x}
%\pair{(y+(-7))+x}{(y-7)+x}
%\pair{(y+(-7))+x}{x+(y+(-7))}
%
%\end{kortit}

%\begin{kortit}{J} %vastaluku ja vähennyslaskun määritelmä
%\pair{x+(-1)}{x-1}
%\pair{-x+(-1)}{-x-1}
%\pair{-x-(-1)}{1-x}
%\pair{1-(-x)}{1+x}
%\pair{-2-x}{-2+(-x)}
%\pair{2-(+x)}{2-x}
%\pair{2-(+7)}{-5}
%\pair{7-(+2)}{5}
%\pair{5-(-7)}{12}
%\pair{-(-x)-y}{x-y}
%\pair{-(-y)}{y}
%\pair{+(-y)}{-y}
%\pair{+(+y)}{+y}
%\pair{-(+y)}{-y}
%\pair{2-(-x)-(-y)}{2+x+y}
%\pair{3-2-1}{3+(-2)+(-1)}
%\pair{3-2-1}{3-2+(-1)}
%\pair{3-2-1}{3+(-2)-1}
%\end{kortit}

%\begin{kortit}{K} %kertolaskuaksioomia yms.
%\pair{2\cdot 3}{3\cdot 2}
%\pair{1(2y-5)}{2y-5}
%\pair{1x}{x1}
%\pair{(x+1)y}{y(x+1)}
%\pair{2(x-1)x}{2x(x-1)}
%\pair{(-1)\cdot(x+1)}{(x+1)\cdot (-1)}
%\pair{(2\cdot(x+1))\cdot y}{2\cdot((x+1)\cdot y)}
%\pair{2x}{x2}
%\pair{(1-x)(x-1)}{(x-1)(1-x)}
%\pair{((1-x)(x-1))(1+x)}{(1-x)((x-1)(1+x))}
%\pair{((1-x)(x-1))(1+x)}{((x-1)(1-x))(1+x)}
%\pair{(x-x)0}{0(x-x)}
%\pair{(x-x)0}{0\cdot0}
%\pair{(x+(-x))0}{(x-x)0}
%\pair{\left[1(x+1)\right](x-1)}{(x+1)(x-1)}
%\pair{\left[1(x+1)\right](x-1)}{1\left[(x+1)(x-1)\right]}
%\pair{\left[1(x+1)\right](x-1)}{1\left[(x-1)(x+1)\right]}
%\pair{\left[1(x+1)\right](x-1)}{1(x-1)(x+1)}
%\end{kortit}

%lisää aksioomia

%ovatko lausekkeet yhtä suuret
%\begin{cards}{D}
%\pair{x\cdot x-x}{(x-1)\cdot x}
%\pair{-(x-2)}{2-x}
%\pair{\dfrac{x}{2}}{\dfrac{3}{6}\cdot x}
%\pair{\dfrac{2}{x}:(2-x)}{\dfrac{2}{2x- x\cdot x}}
%\pair{\dfrac{x+2}{x}}{1+\dfrac{2}{x}}
%\pair{\dfrac{x-x}{1+x}}{0}
%\pair{x+(-2)}{x-2}
%\pair{x-\dfrac{x-1}{2}}{\dfrac{x+1}{2}}
%\pair{(x-1)-(1-x)}{2x-2}
%\pair{\dfrac{x\cdot x-1}{x-1}}{x+1}
%\pair{\dfrac{4}{2x}:\dfrac{1}{x}}{2}
%\pair{x-x\cdot(1-x)}{x\cdot x}
%\pair{\dfrac{xxxx}{xx}}{xx}
%\pair{\dfrac{xxx}{xxxx}}{\dfrac{1}{x}}
%\pair{1-(1-x)}{x}
%\pair{x\cdot \dfrac{1}{x}}{1}
%\pair{\dfrac{\frac{1}{x}}{x}}{\dfrac{1}{x\cdot x}}
%\pair{\dfrac{1}{\frac{1}{x}}}{x}
%\end{cards}
%
%\begin{cards}{L}
%\pair{y=x^3-x}{
%\begin{tikzpicture}[smooth, scale=0.5] %domain=-1.75:1.75, 
%    \draw[very thin,color=gray] (-4,-4) grid (4,4);
%    \draw[->] (-4.2,0) -- (4.2,0) node[right] {$x$};
%    \draw[->] (0,-4.2) -- (0,4.2) node[above] {$y$};
%	\clip (-4,-4) rectangle (4,4);
%    \draw[color=red] plot[id=x] function{x**3-x} node[right] {};
%\end{tikzpicture}
%}
%
%\pair{y=1-x}{
%\begin{tikzpicture}[smooth, scale=0.5] %domain=-1.75:1.75, 
%    \draw[very thin,color=gray] (-4,-4) grid (4,4);
%    \draw[->] (-4.2,0) -- (4.2,0) node[right] {$x$};
%    \draw[->] (0,-4.2) -- (0,4.2) node[above] {$y$};
%	\clip (-4,-4) rectangle (4,4);
%    \draw[color=red] plot[id=x] function{1-x} node[right] {};
%\end{tikzpicture}
%}
%
%\pair{y=x}{
%\begin{tikzpicture}[smooth, scale=0.5] %domain=-1.75:1.75, 
%    \draw[very thin,color=gray] (-4,-4) grid (4,4);
%    \draw[->] (-4.2,0) -- (4.2,0) node[right] {$x$};
%    \draw[->] (0,-4.2) -- (0,4.2) node[above] {$y$};
%	\clip (-4,-4) rectangle (4,4);
%    \draw[color=red] plot[id=x] function{x} node[right] {};
%\end{tikzpicture}
%}
%
%\pair{y=2x}{
%\begin{tikzpicture}[smooth, scale=0.5] %domain=-1.75:1.75, 
%    \draw[very thin,color=gray] (-4,-4) grid (4,4);
%    \draw[->] (-4.2,0) -- (4.2,0) node[right] {$x$};
%    \draw[->] (0,-4.2) -- (0,4.2) node[above] {$y$};
%	\clip (-4,-4) rectangle (4,4);
%    \draw[color=red] plot[id=x] function{2*x} node[right] {};
%\end{tikzpicture}
%}
%
%\pair{y=x^2}{
%\begin{tikzpicture}[smooth, scale=0.5] 
%    \draw[very thin,color=gray] (-4,-4) grid (4,4);
%    \draw[->] (-4.2,0) -- (4.2,0) node[right] {$x$};
%    \draw[->] (0,-4.2) -- (0,4.2) node[above] {$y$};
%	\clip (-4,-4) rectangle (4,4);
%    \draw[color=red] plot[id=x] function{x**2} node[right] {};
%\end{tikzpicture}
%}
%
%\pair{y=x+1}{
%\begin{tikzpicture}[smooth, scale=0.5] %domain=-1.75:1.75, 
%    \draw[very thin,color=gray] (-4,-4) grid (4,4);
%    \draw[->] (-4.2,0) -- (4.2,0) node[right] {$x$};
%    \draw[->] (0,-4.2) -- (0,4.2) node[above] {$y$};
%	\clip (-4,-4) rectangle (4,4);
%    \draw[color=red] plot[id=x] function{x+1} node[right] {};
%\end{tikzpicture}
%}
%
%\pair{y=\lvert x \rvert}{
%\begin{tikzpicture}[ scale=0.5] %domain=-1.75:1.75, 
%    \draw[very thin,color=gray] (-4,-4) grid (4,4);
%    \draw[->] (-4.2,0) -- (4.2,0) node[right] {$x$};
%    \draw[->] (0,-4.2) -- (0,4.2) node[above] {$y$};
%	\clip (-4,-4) rectangle (4,4);
%    \draw[color=red] plot[id=x] function{abs(x)} node[right] {};
%\end{tikzpicture}
%}
%
%\pair{y=\lvert x +1\rvert}{
%\begin{tikzpicture}[scale=0.5] %domain=-1.75:1.75, 
%    \draw[very thin,color=gray] (-4,-4) grid (4,4);
%    \draw[->] (-4.2,0) -- (4.2,0) node[right] {$x$};
%    \draw[->] (0,-4.2) -- (0,4.2) node[above] {$y$};
%	\clip (-4,-4) rectangle (4,4);
%    \draw[color=red] plot[id=x] function{abs(x+1)} node[right] {};
%\end{tikzpicture}
%}
%
%
%%\pairg{y=x^3-x}{x**3-2*x}{2}
%\end{cards}




%\begin{cards}{D} %trigonometric identities
%\pair{\sin^2 x + \cos^2 x}{1}
%\pair{\cos^2 x \sin x + \sin^3 x}{\sin x}
%\end{cards}



%todo: funktio- derivaattafunktioparit kuvaajien avulla

%todo (lopulta) korttipakan generointi niin, että on voinut valita flagit aksioomittain, merkinnöittäin, laskutoimituksittain jne.
%todo: vaihdantalakia ja liitäntälakia!

%\begin{kortit}{F}
%\pair{V(r)=\dfrac{4}{3}\pi r^3}{\mathbb{R}_+ \rightarrow \mathbb{R}_+}
%\pair{v(t)=\dfrac{s}{t}}{\mathbb{R}_+ \rightarrow \mathbb{R}}
%\pair{v(s)=\dfrac{s}{t}}{\mathbb{R} \rightarrow \mathbb{R}}
%\pair{s(t)=vt}{\mathbb{R}_+ \rightarrow \mathbb{R}}
%\pair{s(v)=vt}{\mathbb{R} \rightarrow \mathbb{R}}
%\pair{F(m)=ma}{\mathbb{R}_+ \rightarrow \mathbb{R}}
%\pair{T(l)=2\pi \sqrt{\dfrac{l}{g}}  }{\mathbb{R}_+ \rightarrow \mathbb{R}_+}
%\pair{\rho(m)=\dfrac{m}{V} }{\mathbb{R}_+ \rightarrow \mathbb{R}_+}
%\pair{\rho(V)=\dfrac{m}{V}}{\mathbb{R}_+ \rightarrow \mathbb{R}_+}
%\pair{ N(V)=\rho V g}{\mathbb{R}_+ \rightarrow \mathbb{R}_+}
%\pair{ E(v)=\dfrac{1}{2}mv^2}{\mathbb{R} \rightarrow \mathbb{R}_+}
%\pair{ P(t)=\dfrac{E}{t}}{\mathbb{R}_+ \rightarrow \mathbb{R}_+}
%\pair{F(r)=\gamma \dfrac{m_1m_2}{r^2} }{\mathbb{R}_+ \rightarrow \mathbb{R}_+}
%\pair{ \lambda(m)=\left( R_\mathrm{H} \left(\dfrac{1}{m^2}-\dfrac{1}{n^2}\right) \right)^{-1} $\vfill$ R_\mathrm{H}>0 $\vfill$ m,n\in \mathbb{N} }{\mathbb{N} \rightarrow \mathbb{R}_+}
%\pair{E(m)=mc^2$ \vfill $\mathrm{c=valonnopeus}}{\mathbb{R}_+ \rightarrow \mathbb{R}_+}
%\pair{T_{\frac{1}{2}}(\lambda)=\dfrac{\ln 2}{\lambda} $ \vfill $ \lambda>0}{\mathbb{R}_+ \rightarrow \mathbb{R}_+}
%\pair{N(t)=N_0e^{-\lambda t}$ \vfill $\lambda, N_0>0}{\mathbb{R}_+ \rightarrow \mathbb{R}_+}
%\pair{f(b)=\left(\dfrac{1}{a}+\dfrac{1}{b}\right)^{-1} $\vfill$ a,b\in \mathbb{R}\setminus \lbrace 0 \rbrace}{\mathbb{R}\setminus \lbrace 0 \rbrace \rightarrow \mathbb{R} \setminus \lbrace 0 \rbrace}
%\end{kortit}

