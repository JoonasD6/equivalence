% Muuta tiedostonimen paikalle oman lafkasi logo ja muokkaa tarvittaessa kokoa scale-parametrilla

\newcommand\logo{\includegraphics*[scale=0.4]{laakis-logo.png}} %tasaa logon alareuna lausekekoodin kanssa

%muuta korttien kokoa
%\usepackage[paperwidth=6.00cm, paperheight=6.00cm, left=0.50cm, right=0.50cm, top=0.50cm, bottom=0.50cm]{geometry} %neliökortit
\usepackage[paperwidth=10.5cm, paperheight=7.425cm, left=0.50cm, right=0.50cm, top=0.40cm, bottom=0.40cm]{geometry} %9 kpl A4:lle -kortit

%printtauspdf valmiiksi, eli tehdään eri kappalemäärät pdf:ään valmiiksi, ettei tarvitse tulostinsoftaa

% Aseta korttien taustakuva
	\backgroundsetup{
		scale=1,
		color=black,
		opacity=1.0,
		angle=0,
		contents={
			\includegraphics[width=\paperwidth,height=\paperheight]{example-image}
		}
	}

% Määritellään korttiluokat ja kuvaukset
\newcommand{\descriptionA}{<lyhyesti A-luokan sisältö>}
\newcommand{\descriptionB}{<lyhyesti B-luokan sisältö>}

%0= sanallisesti lukulaskuja? kaksi  miljoonaa plus kolme miljoonaa tjsp.?ä – vielä perustavanlaatuisempi testaa biologisia matemaattisia?
%A = yhteenlasku- ja vähennyslasku numeroilla ja muuttujilla, ei osittelulakia ja välttämätöntä kertolaskua
%B= kertolaskua, keskittyminen laskujärjestykseen ja osittelulakiin
%C = jakolaskua, murtolausekkeiden sieventämistä
%D= potenssi
%E= juuret
%F= trigonometria
%G = eksponenttifunktio ja logaritmifunktio
%H = Kaikkia edellisiä
%I = derivaatta?
%J = integraali?
% kompleksiluvut?
% propositio- ja predikaattilogiikka
% joukko-oppi
%vektorialgebraa!