% !TeX root = main.tex
%todo: cards-> deck
%Korttien lisääminen komennolla \pair{x}{y}, missä x ja y ovat saman asian kaksi eri muotoa.
%\pairm{x}{y} automatically assumes math display mode, no need to add dollar signs
%\pairg{}{}, kun _jälkimmäiseen_ laitetaan kuvaaja %make even simpler to ease addition of cards? Input from text file? separated by ampersand instead of in curly braces?

%korttien määrä kannattaa pitää pienissä erissä 18:ssa ja vähän isommissa 36:ssa; isommat on epärohkaisevan isoja läjiä

%switch sille, että minkä luokan kortit printataan.


%\begin{deck}{M5} %sama kaava eri muodoissa
%\pair{$ ax^2+bx+c=0 $}{$ x=\dfrac{-b\pm\sqrt{b^2-4ac}}{2a} $}
%\pair{$ F=ma$}{$ a=\dfrac{F}{m}$}
%\end{deck}

%vektorit ja niiden lineaarikombinaatiot


%\begin{deck}{C1}%aminohapot; smaller size
%\pair{glysiini}{\chemfig{H_3N-[:30]-[:-30]COOH}}
%\pair{alaniini}{\chemfig{H_3N-[:30](-[:90])-[:-30]COOH}}
%\end{deck}

%mikä voidaan johtaa mistäkin, langan jännitysvoima, pinnantukivoima, ilmanvastus

%\begin{deck}{A2}
%\pair{$11$}{\text{yksitoista}}
%\pair{$110$}{\text{satakymmenen}}
%\pair{$101$}{\text{satayksi}}
%\pair{$111$}{\text{satayksitoista}}
%\pair{$1\,000$}{\text{tuhat}}
%\pair{$1\,100$}{\text{tuhat sata}}
%\pair{$1\,010$}{\text{tuhat kymmenen}}
%\pair{$1\,110$}{\text{tuhat satakymmenen}}
%\pair{$1\,001$}{\text{tuhat yksi}}
%\pair{$1\,011$}{\text{tuhat yksitoista}}
%\pair{$1\,101$}{\text{tuhat satayksi}}
%\pair{$1\,111$}{\text{tuhat satayksitoista}}
%\pair{$11\,111$}{\text{yksitoistatuhatta satayksitoista}}
%\pair{$11\,100$}{\text{yksitoistatuhatta sata}}
%\pair{$11\,010$}{\text{yksitoistatuhatta kymmenen}}
%\pair{$11\,001$}{\text{yksitoistatuhatta yksi}}
%\pair{$10\,111$}{\text{kymmenentuhatta satayksitoista}}
%\pair{$10\,010$}{\text{kymmenentuhatta kymmenen}}
%\end{deck}
%
%\begin{deck}{murto1} %associate egyptian fractions with pie diagrams
%	\pair{$ \dfrac{1}{2} $}{\circfrac{1}{2}}
%	\pair{$ \dfrac{1}{3} $}{\circfrac[25]{1}{3}}
%	\pair{$ \dfrac{1}{4} $}{\circfrac[120]{1}{4}}
%	\pair{$ \dfrac{1}{5} $}{\circfrac[160]{1}{5}}
%	\pair{$ \dfrac{1}{6} $}{\circfrac{1}{6}}
%	\pair{$ \dfrac{1}{7} $}{\circfrac[-30]{1}{7}}
%	\pair{$ \dfrac{1}{8} $}{\circfrac{1}{8}}
%	\pair{$ \dfrac{1}{9} $}{\circfrac[30]{1}{9}}
%	\pair{$ \dfrac{1}{10} $}{\circfrac{1}{10}}
%\end{deck}
%
%\begin{deck}{murto2} %associate egyptian fractions with rectangular grid diagrams %originally just square here and rectangles later
%	\pair{$ \dfrac{1}{4} $}{\rectfrac{1}{2}}
%	\pair{$ \dfrac{1}{9} $}{\rectfrac{1}{3}}
%	\pair{$ \dfrac{1}{16} $}{\rectfrac{1}{4}}
%	\pair{$ \dfrac{1}{25} $}{\rectfrac[-0.8]{1}{5}}
%	\pair{$ \dfrac{1}{36} $}{\rectfrac[0.8]{1}{6}}
%	\pair{$\dfrac{1}{12}$}{
%	\adjustbox{valign=m}{\begin{tikzpicture}[scale=0.5]
%	\fill[color=gray] (0,0) rectangle (1,1);
%	\draw (0,0) grid (3,4);
%	\end{tikzpicture}}}
%	\pair{$\dfrac{1}{15}$}{
%	\adjustbox{valign=m}{\begin{tikzpicture}[scale=0.5,rotate=90]
%	\fill[color=gray] (0,0) rectangle (1,1);
%	\draw (0,0) grid (3,5);
%	\end{tikzpicture}}}
%	\pair{$ \dfrac{1}{6} $}{	\adjustbox{valign=m}{\begin{tikzpicture}[scale=-0.5]
%		\fill[color=gray] (0,0) rectangle (1,2);
%		\draw[yscale=2] (0,0) grid (3,2);
%		\end{tikzpicture}}}
%	\pair{$ \dfrac{1}{8} $}{	\adjustbox{valign=m}{\begin{tikzpicture}[scale=0.5]
%		\fill[color=gray] (0,0) rectangle (1,2);
%		\draw[yscale=2] (0,0) grid (4,2);
%		\end{tikzpicture}}}
%	
%\end{deck}


%\begin{deck}{yhdistetyt1} %prime factorisation of natural numbers with power expressions
%	\pairm{4}{2\cdot2}
%	\pairm{6}{2\cdot3}
%	\pairm{8}{2\cdot2\cdot2}
%	\pairm{9}{3\cdot 3}
%	\pairm{10}{2\cdot5}
%	\pairm{12}{2\cdot2\cdot 3}
%	\pairm{14}{2\cdot7}
%	\pairm{15}{3\cdot5}
%	\pairm{16}{2\cdot2\cdot 2\cdot2}
%	\pairm{18}{2\cdot9}
%	\pairm{20}{2\cdot2\cdot 5}
%	\pairm{21}{3\cdot7}
%	\pairm{22}{2\cdot11}
%	\pairm{24}{2\cdot2\cdot2\cdot3}
%	\pairm{25}{5\cdot5}
%	\pairm{26}{2\cdot13}
%	\pairm{27}{3\cdot3\cdot3}
%	\pairm{28}{2\cdot2\cdot7}
%\end{deck}


%simppelit osittelulait ilman miinuksia ja vain kokonaislukukertoimia
%\begin{deck}{dist1}
%\pairm{x+x}{(1+1)x}
%\pairm{8y+7y}{(8+7)y}
%\pairm{4x+x}{(4+1)x}
%\pairm{9x+10x}{19x}
%\pairm{4(x+1)}{4x+4}
%\pairm{2(2+y)}{4+2y}
%\pairm{y(y+1)}{y^2+y}
%\pairm{6(3+y)}{6\cdot 3 + 6y}
%\pairm{2(2y+1)}{4y+2}
%\end{deck}

%\begin{deck}{murto3} %associate proper fractions with pie diagrams
%	\pair{$ \dfrac{2}{3} $}{\circfrac{2}{3}}
%	\pair{$ \dfrac{3}{4} $}{\circfrac[25]{3}{4}}
%	\pair{$ \dfrac{4}{5} $}{\circfrac[120]{4}{5}}
%	\pair{$ \dfrac{5}{6} $}{\circfrac[160]{5}{6}}
%	\pair{$ \dfrac{6}{7} $}{\circfrac{6}{7}}
%	\pair{$ \dfrac{2}{5} $}{\circfrac[-30]{2}{7}}
%	\pair{$ \dfrac{3}{7} $}{\circfrac{3}{7}}
%	\pair{$ \dfrac{4}{9} $}{\circfrac[30]{4}{9}}
%	\pair{$ \dfrac{5}{11} $}{\circfrac{5}{11}}
%\end{deck}
%
%\begin{deck}{murto4} %sums and differences of true fractions (of varying numerators; only coprime) shown as rectangular diagrams of mutually equal size
%	\pairm{\dfrac{1}{2}}{
%		\adjustbox{valign=m}{\begin{tikzpicture}[scale=0.5]
%				\fill[color=gray] (1,0) rectangle (3,2);
%				\draw (0,0) grid (3,4);
%		\end{tikzpicture}}+
%		\adjustbox{valign=m}{\begin{tikzpicture}[scale=0.5]
%				\fill[color=gray] (0,0) rectangle (1,2);
%				\draw[yscale=2] (0,0) grid (3,2);
%	\end{tikzpicture}}}
%	\pairm{\dfrac{2}{3}}{
%		\adjustbox{valign=m}{\begin{tikzpicture}[scale=0.5]
%				\fill[color=gray] (2,0) rectangle (3,2);
%				\draw (0,0) grid (3,4);
%		\end{tikzpicture}}+
%		\adjustbox{valign=m}{\begin{tikzpicture}[scale=0.5]
%				\fill[color=gray] (0,0) rectangle (3,2);
%				\draw[yscale=2] (0,0) grid (3,2);
%	\end{tikzpicture}}}
%	\pairm{\dfrac{1}{2}}{
%		\adjustbox{valign=m}{\begin{tikzpicture}[scale=0.5]
%				\fill[color=gray] (2,0) rectangle (3,2);
%				\draw (0,0) grid (3,4);
%		\end{tikzpicture}}+
%		\adjustbox{valign=m}{\begin{tikzpicture}[scale=0.5]
%				\fill[color=gray] (0,0) rectangle (2,2);
%				\draw[yscale=2] (0,0) grid (3,2);
%	\end{tikzpicture}}}
%	\pairm{\dfrac{13}{16}}{
%		\adjustbox{valign=m}{\begin{tikzpicture}[scale=0.5]
%				\fill[color=gray] (1,0) rectangle (4,3);
%				\draw (0,0) grid (4,4);
%		\end{tikzpicture}} +
%		\adjustbox{valign=m}{\begin{tikzpicture}[scale=0.5]
%				\fill[color=gray] (2,0) rectangle (4,2);
%				\draw[scale=2] (0,0) grid (2,2);
%		\end{tikzpicture}} 
%	}
%	\pairm{\dfrac{1}{6}}{
%		\adjustbox{valign=m}{\begin{tikzpicture}[scale=0.5]
%				\fill[color=gray] (0,0) rectangle (2,2);
%				\draw[yscale=2] (0,0) grid (3,2);
%		\end{tikzpicture}}-
%		\adjustbox{valign=m}{\begin{tikzpicture}[scale=0.5]
%				\fill[color=gray] (1,0) rectangle (3,1);
%				\draw (0,0) grid (3,4);
%		\end{tikzpicture}}
%	}
%	\pairm{\dfrac{1}{2}}{ 
%		\adjustbox{valign=m}{\begin{tikzpicture}[scale=0.5]
%				\fill[color=gray] (0,0) rectangle (2,4);
%				\draw[yscale=1, xscale=1] (0,0) grid (3,4);
%		\end{tikzpicture}}-
%		\adjustbox{valign=m}{\begin{tikzpicture}[scale=0.5]
%				\fill[color=gray] (1,0) rectangle (3,1);
%				\draw (0,0) grid (3,4);
%		\end{tikzpicture}}
%	}
%	
%	\pairm{\dfrac{4}{6}}{\circfrac{1}{2} + \circfrac{1}{6}}
%	\pairm{\dfrac{2}{6}}{\circfrac{1}{2} - \circfrac{1}{6}}
%	\pairm{\dfrac{2}{7}}{\circfrac{1}{7} + \circfrac{2}{14}}
%	\pairm{\dfrac{1}{14}}{\circfrac{1}{7} - \circfrac{1}{14}}
%	\pairm{\dfrac{1}{12}}{\circfrac{1}{6} - \circfrac{1}{12}}
%	\pairm{\dfrac{5}{6}}{\circfrac{1}{2} + \circfrac{1}{3}}
%	\pairm{\dfrac{2}{2}}{\circfrac{1}{2} + \circfrac{1}{2}}
%	\pairm{\dfrac{3}{2}}{\circfrac{3}{4} + \circfrac{3}{4}}
%	\pairm{\dfrac{1}{4}}{\circfrac{1}{2} - \circfrac{1}{4}}
%	\pairm{\dfrac{1}{8}}{\circfrac{1}{4} - \circfrac{1}{8}}
%	\pairm{\dfrac{1}{10}}{\circfrac{1}{5} - \circfrac{1}{10}}
%	\pairm{\dfrac{1}{3}}{\circfrac{1}{1} - \circfrac{2}{3}}
%\end{deck}

%sit miinuksilla binomin sisällä
%sit miinuskertoimilla
%sit binomi kertaa binomi...

%murtolukujen yhteenlasku tunnetuilla luvuilla
%\begin{deck}{murto5}
%\pairm{\frac{1}{1}+\frac{1}{1}}{\frac{2}{1}}
%\pairm{\frac{1}{2}+\frac{1}{1}}{\frac{3}{2}}
%\pairm{\frac{1}{4}+\frac{1}{2}}{\frac{3}{4}}
%\pairm{\frac{1}{3}+\frac{1}{9}}{\frac{4}{9}}
%\pairm{\frac{1}{3}+\frac{1}{4}}{\frac{7}{12}}
%\pairm{\frac{1}{4}+\frac{1}{5}}{\frac{9}{20}}
%\pairm{\frac{1}{5}+\frac{1}{6}}{\frac{11}{30}}
%\pairm{\frac{1}{2}+\frac{1}{7}}{\frac{9}{14}}
%\pairm{\frac{1}{3}+\frac{1}{7}}{\frac{10}{21}}
%\end{deck}


%\begin{deck}{A2} % sum, subtraction, additive inverse, definition of subtraction with variables + addition of variables
%	\pairm{5+(-x)}{5-x}
%	\pairm{2+(+x)}{2+x}
%	\pairm{x-y-x}{-y}
%	\pairm{x+x-x+x}{2x}
%	\pairm{-x}{x(1-2)}
%	\pairm{-x+(x+y)}{y}
%	\pairm{2-x-2}{-x}
%	\pairm{x-2-x}{-2}
%	\pairm{xy-2xy-yx+(-yx)}{-3xy}
%	\pairm{(x+y)-x}{x+(y-x)}
%	\pairm{2-2x-(2x)}{2-4x}
%	\pairm{(x+y)-x+(-y)}{0}
%	\pairm{(-x)-(-x)}{0}
%	\pairm{x+(-1)}{x-1}
%	\pairm{-x+(-1)}{-x-1}
%	\pairm{-x-(-1)}{1-x}
%	\pairm{1-(-x)}{1+x}
%	\pairm{-2-x}{-2+(-x)}
%\end{deck}

%\pairm{2-(+x)}{2-x}
%\pairm{2-(+7)}{-5}
%\pairm{7-(+2)}{5}
%\pairm{5-(-7)}{12}
%\pairm{-(-x)-y}{x-y}
%\pairm{-(-y)}{y}
%\pairm{+(-y)}{-y}
%\pairm{+(+y)}{+y}
%\pairm{-(+y)}{-y}
%\pairm{2-(-x)-(-y)}{2+x+y}
%\pairm{3-2-1}{3+(-2)+(-1)}
%\pairm{3-2-1}{3-2+(-1)}
%\pairm{3-2-1}{3+(-2)-1}

%\begin{deck}{A4} %products with known values, minus signs (no distributive law!)
%\pairm{(-1)(-2)}{2}
%\pairm{(-1)(-2)(-1)}{-2}
%\pairm{[(-2)(-1)]\cdot 2}{4}
%\end{deck}

%\begin{deck}{A3} %products with variables, minus signs, natural numbers (no distributive law!)
%	\pairm{(-1)(-2)}{2}
%	\pairm{(-1)(-2)(-1)}{-2}
%	\pairm{[(-2)(-1)]\cdot 2}{4}
%	\pairm{-[(-z)\cdot (-y)]}{-(yz)}
%	\pairm{(-3)[x(-y)]}{3xy}
%	\pairm{2t-3t(x-t)\}{2t}
%	\pairm{-(2z)\cdot(-(xy))}{2zyx}
%	\pairm{-[(-x)\cdot y]}{yx}
%	\pairm{y\cdot(-x)}{-(xy)}
%	\pairm{(-2)[(-x)\cdot y]}{2xy}
%	\pairm{-(-x)\cdot [-(-y)]}{xy}
%	\pairm{(-x)(-y)(-z)}{-xyz}
%	\pairm{(-2x)(-2y)(-2z)}{-8xyz}
%	\pairm{0\cdot xy}{0}
%	\pairm{-(-x)}{x}
%	\pairm{-(-xt)}{xt}
%	\pairm{(-2x)t}{-2xt}
%	\pairm{(-2)\cdot (-x)(-y)}{-2xy}
%\end{deck}


%murtolukujen vähennyslasku tunnetuilla luvuilla
%tunnetuilla kolme murtolukua yhteen...


%murtolukujen yhteen- ja vähennyslasku tuntemattomilla (yksi symboli ja isompi lauseke) osoittajassa %pitää jakaa pienempiin osiin...
%\begin{deck}{murto6}
%\pairm{\frac{t}{2}+\frac{t}{2}}{t}
%\pairm{\frac{t}{3}-\frac{t}{3}}{0}
%\pairm{\frac{t}{4}+\frac{1}{2}}{\frac{t+2}{4}}
%\pairm{\frac{1}{3}+\frac{t}{6}}{\frac{2+t}{6}}
%\pairm{\frac{t}{3}+\frac{1}{4}}{\frac{4t+3}{12}}
%\pairm{\frac{t+1}{3}+\frac{1}{4}}{\frac{4t+7}{12}}
%\pairm{\frac{t}{4}+\frac{t}{5}}{\frac{9t}{20}}
%\pairm{\frac{t+2}{5}+\frac{1}{6}}{\frac{6t+17}{30}}
%\pairm{\frac{t+1}{2}+\frac{t+1}{4}}{\frac{3t+3}{4}}
%\pairm{\frac{t}{3}+\frac{t+1}{7}}{\frac{10t+3}{21}}
%\pairm{t+\frac{1}{7}}{\frac{7t+1}{7}}
%\pairm{t+\frac{t}{7}}{\frac{8t}{7}}
%\pairm{2+\frac{t}{7}}{\frac{14+t}{7}}
%\pairm{3+\frac{t+2}{4}}{\frac{t+14}{4}}
%\pairm{t+\frac{t+2}{4}}{\frac{5t+2}{4}}
%\pairm{t+5+\frac{1}{4}}{\frac{4t+21}{4}}
%\pairm{2t+1+\frac{2t+1}{9}}{\frac{20t+10}{9}}
%\pairm{t-\frac{t+2}{4}}{\frac{3t-2}{4}}
%\end{deck}


%murtolukujen yhteen- ja vähennyslasku tuntemattomilla (yksi symboli ja isompi lauseke) nimittäjässä
%\begin{deck}{murto7}

%\end{deck}


%\begin{deck}{murto3} %multiplication of true fractions shown as rectangular diagrams of equal size
%\pairm{\dfrac{1}{6}}{
%	\adjustbox{valign=m}{\begin{tikzpicture}[scale=0.5]
%		\fill[color=gray] (0,0) rectangle (2*0.75,2);
%		\draw[yscale=0.667, xscale=0.75] (0,0) grid (4,6);
%		\end{tikzpicture}}\cdot
%	\adjustbox{valign=m}{\begin{tikzpicture}[scale=0.5]
%		\fill[color=gray] (1,0) rectangle (3,1);
%		\draw (0,0) grid (3,4);
%		\end{tikzpicture}}
%}
%\pairm{\dfrac{2}{6}}{\circlefrac{1}{2} \cdot \circfrac{4}{6}} %voi siirtää muualle
%\end{deck}

%\begin{deck}{set2} %sets and venn diagrams
%	\pairm{}{}
%\end{deck}

%\begin{deck}{set1} %set arithmetic without real intervals or number sets
%%	\pairm{\lbrace 1, 2, 3\rbrace \cap \mathbb{N}}{\lbrace 1, 2, 3\rbrace}
%%	\pairm{\lbrace 1, 2, 3\rbrace \cup \mathbb{N}}{\mathbb{N}}
%%	\pairm{\lbrace \text{omena}, \text{banaani}, \text{päärynä}\rbrace \cup \lbrace \text{banaani} \rbrace}{\lbrace \text{omena}, \text{banaani}, \text{päärynä}\rbrace}
%%	\pairm{\lbrace \text{omena}, \text{banaani}, \text{päärynä}\rbrace \cap \lbrace \text{banaani} \rbrace}{\lbrace\text{banaani}\rbrace}
%	\pairm{\lbrace \text{omena}, \text{banaani}, \text{päärynä}\rbrace \setminus \lbrace \text{banaani} \rbrace}{\lbrace \text{omena}, \text{päärynä}\rbrace}
%%	\pairm{\lbrace 1,2 \rbrace \cap \lbrace 2, 4 \rbrace}{ \lbrace 2 \rbrace}
%%	\pairm{\lbrace 1,2 \rbrace \cup \lbrace 2, 4 \rbrace}{ \lbrace 1, 2, 4 \rbrace}
%	\pairm{\lbrace 1,2 \rbrace \setminus \lbrace 2, 4 \rbrace}{ \lbrace 1\rbrace}
%%	\pairm{\lbrace \heartsuit \rbrace \cap \lbrace \spadesuit \rbrace }{\emptyset}
%%	\pairm{\lbrace \heartsuit \rbrace \cup \lbrace \spadesuit \rbrace }{\lbrace \heartsuit, \spadesuit \rbrace}
%	\pairm{\lbrace \heartsuit \rbrace \setminus \lbrace \spadesuit \rbrace }{\lbrace \heartsuit \rbrace}
%%	\pairm{\mathbb{R} \setminus (\mathbb{R}_- \cup \lbrace0\rbrace )}{\mathbb{R}_+}
%	\pairm{\mathbb{R}\setminus \mathbb{Q}}{\text{irrationaalilukujen joukko}}
%	\pairm{\mathbb{R}\setminus \mathbb{R}_-}{\text{epänegatiivisten lukujen joukko}}
%	\pairm{\mathbb{N}\setminus \mathbb{Q}}{\text{tyhjä joukko (ei sisällä alkioita)}}
%	\pairm{\mathbb{Z}\setminus \mathbb{N}}{\text{epäpositiiviset kokonaisluvut}}
%	\pairm{\mathbb{Z} \setminus \mathbb{Z}_-}{\text{epänegatiiviset kokonaisluvut}}
%	\pairm{\mathbb{Z} \setminus \mathbb{Z}_- \setminus \lbrace0\rbrace}{\text{positiiviset kokonaisluvut}}
%	\pairm{\mathbb{Z} \setminus \mathbb{Z}_- \setminus \mathbb{Z}_+}{\lbrace 0\rbrace}
%%	\pairm{\mathbb{R} \cap \mathbb{R}_-}{\mathbb{R}_-}
%%	\pairm{\mathbb{R}_+ \cup \lbrace 0 \rbrace}{\text{epänegatiiviset luvut}}
%%	\pairm{\mathbb{R}\cup \mathbb{Q}}{\mathbb{R}}
%%	\pairm{\mathbb{R}\setminus \lbrace 3 \rbrace}{]-\infty, 3[ \cup ]3, \infty[}
%	\pairm{\mathbb{N}}{\lbrace1, 2, 3, ...\rbrace}
%	\pairm{\lbrace 2, 3, 5, 7, 11, ...\rbrace}{\text{alkulukujen joukko}}
%	\pairm{\lbrace k\in \mathbb{Z}_+ \mid \text{k on parillinen}\rbrace  }{\lbrace 2, 4, 6...\rbrace}
%	\pairm{\lbrace x\in \mathbb{R} \mid0<x\leq2\rbrace }{]0, 2]}
%	\pairm{0\leq x\leq2}{x\in[0, 2]}
%	\pairm{x>0}{x\in ]0, \infty[}
%	\pairm{x\geq 0}{x\in[0, \infty[}
%	\pairm{x\leq 0}{x\in]-\infty, 0]}
%\end{deck}
%set arithmetic with number sets
%set arithmetic with real intervals %esim. raja ja pois diskreetit
%sets with Venn diagrams
%tehdään toinen sarja, jossa lisää reaalilukuvälejä ja erilaisia tapoja esittää joukot ja kahden joukko-operaation kortteja

%kvanttorit

%\begin{deck}{log1} %propositional logic connectives
%	\pair{$P$ ja $Q$}{$P \wedge Q$}
%	\pair{sekä $P$ että $Q$}{$P \wedge Q$}
%	\pair{Ei $ P $ eikä $ Q $}{$ \neg P \wedge \neg Q $}
%	\pair{$P$ tai $Q$}{$P \vee Q$}
%	\pair{$P$ tai $Q$ tai molemmat}{$P \vee Q$}
%	\pair{joko $P$ tai $Q$ tai molemmat}{$P \vee Q$}
%	\pair{$P$, mutta $Q$}{$P \wedge Q$}
%	\pair{$P$, vaikka $Q$}{$P \wedge Q$}
%	\pair{Jos $P$, niin $Q$}{$P \Rightarrow Q$}
%	\pair{$P$, jos $Q$}{$Q \Rightarrow P$}
%	\pair{$P$ vain, jos $Q$}{$P \Rightarrow Q$}
%	\pair{$P$ aina sillä ehdolla, että $Q$}{$Q \Rightarrow P$}
%	\pair{$P$, jos ja vain jos $Q$}{$P \Leftrightarrow Q$}
%	\pair{$P$ aina, kun $Q$}{$Q \Rightarrow P$}
%	\pair{$P$ on välttämätön ehto $Q$:lle}{$Q \Rightarrow P$}
%	\pair{$P$ on riittävä ehto $Q$:lle}{$P \Rightarrow Q$}
%	\pair{$P$ on riittävä ja välttämätön ehto $ Q $:lle}{$P \iff Q$}
%	\pair{jos ei $P$, niin $Q$}{$\neg P \implies Q$}
%\end{deck}

%\begin{deck}{log2} %propositional logic 2
%	\pairm{m}{m}
%\end{deck}

%\begin{deck}{bio1}
%	\pair{osteoblasti}{rakentaa luukudosta}
%	\pair{osteoklasti}{hajottaa luukudosta}
%	\pair{adiposyytti}{varastoi rasvaa}
%	\pair{neuroni}{välittää signaaleja hermostossa}
%	\pair{erytrosyytti}{kuljettaa happea veressä}
%	\pair{granulosyytti}{}
%	\pair{leukosyytti}{}
%	\pair{lymfosyytti}{}
%	\pair{maljasolu?}{}
%	\pair{Sertolin solu}{}
%	\pair{gliasolu}{}
%	\pair{}{}
%	\pair{}{}
%	\pair{}{}
%	\pair{}{}
%	\pair{}{}
%	\pair{}{}
%	\pair{}{}
%\end{deck}


%\begin{deck}{yhdistetyt2} %larger arbitrary natural numbers with power-like prime factorisation
%
%\end{deck}
%
%\begin{deck}{A10} %multiplication of fractions shown as diagrams
%	\pair{}{}
%\end{deck}

%\begin{deck}{A1} %sum, subtraction, additive inverse, definition of subtraction with known integers
%
%\pairm{-2-(-5)}{3}
%\pairm{5-(-2)}{7}
%\pairm{2(6-1)}{10}
%
%\end{deck}


%\begin{deck}{B1} %cancellation with only numbers
%	
%\pair{\dfrac{15}{9}}{\dfrac{5}{3}}
%\pair{\dfrac{30}{4}}{\dfrac{15}{2}}
%\pair{\dfrac{2\cdot5\cdot 7}{10\cdot 7 \cdot 8}}{\dfrac{1}{8}}
%\pair{\dfrac{1}{2}\cdot \dfrac{2}{3}\cdot \dfrac{3}{4}}{\dfrac{1}{4}}
%\pair{\dfrac{-570}{9}}{\dfrac{190}{-3}}

%
%\end{deck}

%\begin{deck}{B2} %cancellation with variables
%	
%\pair{\dfrac{x}{y}:\dfrac{x^2}{y^2}:\dfrac{x^3}{y^3}}{\dfrac{y^4}{x^4}}
%\end{deck}

%\begin{deck}{C} %equation plots
%\pairg{y=x^2}{x**2}
%\pairg{y=x^2}{1/x}
%\end{deck}

%\begin{deck}{Ax} %definition of subtraction as sum, commutativity
%\pairm{2-2x}{-2x+2}
%\pairm{-x-(-\pi)}{\pi-x}
%\end{deck}
%
%\begin{deck}{A4} %Verbal instructions for arithmetic expressions, incl. additive inverse but no quotient or multiplicative inverse
%\pair{lukujen $ x $ ja $ y $ summa}{$ x+y $}
%\pair{lukujen $ x $ ja $ y $ erotus}{$ x-y $}
%\pair{lukujen $ x $ ja $ y $ tulo}{$ xy $}
%\pair{lukujen $ x $ ja $ y $ tulon vastaluvun vastaluku}{$ -[-(xy)] $}
%\pair{lukujen $ x $ ja $ y $ summan vastaluku}{$ -(x+y) $}
%\pair{lukujen $ x $ ja $ y $ erotuksen vastaluku}{$ -(x-y) $}
%\pair{lukujen $ x $ ja $ y $ tulon vastaluku}{$ -(xy) $}
%\pair{lukujen $ x $ ja $ y $ vastalukujen summa}{$ (-x)+(-y)$}
%\pair{lukujen $ x $ ja $ y $ vastalukujen erotus}{$ (-x)-(-y)$}
%\pair{lukujen $ x $ ja $ y $ vastalukujen tulo}{$ (-x)(-y)$}
%\pair{$ x $:n vastaluvun ja $ y $:n summa}{$ -x+y $}
%\pair{$ x $:n vastaluvun ja $ y $:n erotus}{$ -x-y $}
%\pair{$ x $:n vastaluvun ja $ y $:n tulo}{$ -x\cdot y $}
%\pair{luvun $ x $ sekä $ y $:n vastaluvun summa}{$ x+(-y)$}
%\pair{luvun $ x $ sekä $ y $:n vastaluvun summan vastaluku}{$ -[x+(-y)]$}
%\pair{luvun $ x $ sekä $ y $:n vastaluvun erotus}{$ x-(-y)$}
%\pair{luvun $ x $ sekä $ y $:n vastaluvun erotuksen vastaluku}{$ -[x-(-y)]$}
%\pair{luvun $ x $ sekä $ y $:n vastaluvun tulo}{$ x(-y)$}
%\end{deck}
%
%\begin{deck}{A5} %Verbal instructions for arithmetic expressions, incl. quotient and multiplicative inverse
%	\pair{lukujen $ x $ ja $ y $ osamäärä}{$ \dfrac{x}{y} $}
%	\pair{lukujen $ x $ ja $ y $ vastalukujen osamäärä}{$ \dfrac{-x}{-y} $}
%	\pair{lukujen $ x $ ja $ y $ osamäärän vastaluku}{$ - \dfrac{x}{y} $}
%	\pair{lukujen $ x $ ja $ y $ vastalukujen osamäärän vastaluku}{$-\dfrac{-x}{-y} $}
%	\pair{lukujen $ x $ ja $ y $ käänteislukujen summa}{$ \dfrac{1}{x}+\dfrac{1}{y} $}
%	\pair{lukujen $ x $ ja $ y $ käänteislukujen erotus}{$\dfrac{1}{x}-\dfrac{1}{y}$}
%	\pair{lukujen $ x $ ja $ y $ osamäärän käänteisluku}{${\left(\dfrac{x}{y}\right)}^{-1} $}
%	\pair{lukujen $ x $ ja $ y $ vastalukujen summan käänteisluku}{$ \left( (-x)+(-y)\right)^{-1}$}
%	\pair{lukujen $ x $ ja $ y $ vastalukujen erotuksen käänteisluku}{$ \left( (-x)-(-y)\right)^{-1}$}
%	\pair{$ x $:n vastaluvun ja $ y $:n käänteisluvun summa}{$ -x+\frac{1}{y}$}
%	\pair{$ x $:n käänteisluvun ja $ y $:n vastaluvun summa}{$ \frac{1}{x}+(-y) $}
%	\pair{$ x $:n käänteisluvun ja $ y $:n vastaluvun tulo}{$\dfrac{1}{x}\cdot(-y)$}
%	\pair{lukujen $ x $ ja $ y $:n käänteislukujen tulo}{$ \dfrac{1}{x}\cdot\dfrac{1}{y} $}
%	\pair{lukujen $ x $ ja $ y $ käänteislukujen osamäärä}{$ \dfrac{1}{x}:\dfrac{1}{y} $}
%	\pair{lukujen $ x $ ja $ y $ käänteislukujen summan käänteisluku}{$ \dfrac{1}{\frac{1}{x}+\frac{1}{y}}$}
%	\pair{lukujen $ x $ ja $ y $ käänteislukujen erotuksen käänteisluku}{$ \dfrac{1}{\frac{1}{x}-\frac{1}{y}}$}
%	\pair{lukujen $ x $ ja $ y $ vastalukujen käänteislukujen summa}{$\dfrac{1}{-x}+\dfrac{1}{-y}$}
%	\pair{lukujen $ x $ ja $ y $ vastalukujen käänteislukujen erotus}{$\dfrac{1}{-x}-\dfrac{1}{-y}$}
%\end{deck}


%potenssien tulon ... %tulon potenssin...
%– kortit: annettu yhtälö ja operaatio, yhdistä seuraavaan vaiheeseen
%– matikantehtäväksi sanallisia selityksiä: “mikä on lausekkeessa osoittajan ja nimittäjän yhteinen tekijä?” jne.


%\begin{kortit}{H} %multiplication notation and commutativity
%\pair{9\cdot x}{9x}
%\pair{x\cdot x}{xx}
%\end{kortit}
%
%\begin{deck}{I} %distributive law of whole numbers including minus sign in front of parentheses (no powers or rational expressions)
%\pairm{x(a-1)}{ax-x}
%\pairm{a(x-1)}{ax-a}
%\pairm{a(1-x)}{a-ax} 
%\pairm{x(1-a)}{x-ax}
%\pairm{x-2(2x+4)}{-3x-8}
%\pairm{x+x+x}{3\cdot x}
%\pairm{2(2x-4)-x}{3x-8}
%\pairm{-(1-x)-(1-x)}{-2(1-x)}
%\pairm{-(2-y)}{y-2}
%\pairm{-2(x+1)}{-2x-2}
%\pairm{-2(x-1)}{-2x+2}
%\pairm{-(-2+y)}{2-y}
%\pairm{2t\cdot 0-3t(x-1)}{-3tx+3t}
%\pairm{x+a(1+x)}{a+x(a+1)}
%\pairm{2y(x-1)}{2xy-2y}
%\pairm{2x(y-1)}{2xy-2x}
%\pairm{x(y+1)-y(x+1)}{x-y}
%\pairm{-(x-2)-(x-2)}{-2(x-2)}
%\end{deck}


%order of operations for whole numbers (no variables, no powers)
%\pair{3-3\cdot3}{-6}
%\pair{3\cdot3-3}{6}
%\pair{(3-3)\cdot3}{0}

%
%\begin{deck}{E} %lisää määrittelyjoukot?



%\begin{deck}{M6}%assign domain to expression
%\pair{$x^2$}{$ x \in \mathbb{R} $}
%\pair{$x^2-1$}{$ x\in \mathbb{R} $}
%\end{deck}




%\begin{deck}{M} %pari, muutama aksioomaa ja laskusääntöä, ei käänteislukua tai osittelulakia, kertolasku luonnollisilla
%
%
%\end{deck}


%kertolaskuaksioomia \pair{t(1-2)x}{-tx}

%\pair{\dfrac{x-1}{1-x}}{-1} %B:tä jo?

%\pairm{x+(-2)}{x-2}
%paljon murtolukuja eri muodoissa
%
%kertomerkki -yms. merkitsemiskäytännöt, selvä ero pystyyn kirjoitetutlle ja kursiiville
% yksikköpyörittelyä
%
%

%

%
%
%
%\kB{\dfrac{5}{10}}{\dfrac{6}{10}}
%
%\kB{\dfrac{2x}{x}}{\dfrac{-4x}{2x}}
%
%\setcounter{lauseke}{1}
%



%random osittelulakeja
%\pairm{x\cdot x-x}{(x-1)\cdot x}
%\pairm{-(x-2)}{2-x}
%\pairm{(x-1)-(1-x)}{2x-2}
%\pairm{1-(1-x)}{x}
%\pairm{x-x\cdot(1-x)}{x\cdot x}

 %+murtoluvusta desimaaliiin jne. %\pairm{1\frac{1}{2}}{1+\frac{1}{1}}

%yhteenlaskuaksioomia
%\begin{deck}{J}
%\pairm{2+3}{3+2}
%\pairm{x^2+3}{3+x^2}
%\pairm{(-1)+3}{3+(-1)}
%\pairm{(1+2)+3}{3+(1+2)}
%\pairm{(a+b)+c}{(b+a)+c}
%\pairm{(a+b)+c}{a+(b+c)}
%\pairm{(a+b)+c}{c+(a+b)}
%\pairm{-a+b}{b+(-a)}
%\pairm{\dfrac{1+x}{x}}{\dfrac{x+1}{x}}
%\pairm{-\pi+x^5}{x^5+(-\pi)}
%\pairm{2\sin(x)+\sqrt{y}}{\sqrt{y}+2\sin(x)}
%\pairm{b+(-b)}{b-b}
%\pairm{b+(-b)}{0}
%\pairm{x+(-y)}{x-y}
%\pairm{(y+(-7))+x}{y+((-7)+x)}
%\pairm{(y+(-7))+x}{((-7)+y)+x}
%\pairm{(y+(-7))+x}{(y-7)+x}
%\pairm{(y+(-7))+x}{x+(y+(-7))}
%\end{deck}

%\begin{deck}{K} %kertolaskuaksioomia %VÄHENNÄ TOISTOA
%\pairm{2\cdot 3}{3\cdot 2}
%\pairm{1(2y-5)}{2y-5}
%\pairm{1x}{x1}
%\pairm{(x+1)y}{y(x+1)}
%\pairm{2(x-1)x}{2x(x-1)}
%\pairm{(-1)\cdot(x+1)}{(x+1)\cdot (-1)}
%\pairm{(2\cdot(x+1))\cdot y}{2\cdot((x+1)\cdot y)}
%\pairm{2x}{x\cdot2}
%\pairm{(1-x)(x-1)}{(x-1)(1-x)}
%\pairm{((1-x)(x-1))(1+x)}{(1-x)((x-1)(1+x))}
%\pairm{((1-x)(x-1))(1+x)}{((x-1)(1-x))(1+x)}
%\pairm{(x-x)\cdot0}{0\cdot(x-x)}
%\pairm{(x-x)\cdot0}{0\cdot0}
%\pairm{(x+(-x))\cdot0}{(x-x)\cdot0}
%\pairm{\left[1(x+1)\right](x-1)}{(x+1)(x-1)}
%\pairm{\left[1(x+1)\right](x-1)}{1\left[(x+1)(x-1)\right]}
%\pairm{\left[1(x+1)\right](x-1)}{1\left[(x-1)(x+1)\right]}
%\pairm{\left[1(x+1)\right](x-1)}{1(x-1)(x+1)}
%\end{deck}

%lisää aksioomia


%\begin{deck}{L} %sievennä murtolauseke 1
%	\pairm{\dfrac{45}{5}}{9}
%	\pairm{\dfrac{65}{39}}{\dfrac{5}{3}}
%	\pairm{\dfrac{3k}{k}}{3}
%	\pairm{\dfrac{1}{3}k}{\dfrac{k}{3}}
%	\pairm{\dfrac{k}{3k}}{\dfrac{1}{3}}
%	\pairm{\dfrac{5}{6}+\dfrac{6}{5}}{\dfrac{61}{30}}
%	\pairm{\dfrac{5}{6}-\dfrac{6}{5}}{-\dfrac{11}{30}}
%	\pairm{\dfrac{5}{6}\cdot\dfrac{6}{5}}{1}
%	\pairm{\dfrac{5}{6}:\dfrac{6}{5}}{\dfrac{25}{36}}
%	\pairm{\dfrac{5}{n}+\dfrac{1}{5}}{\dfrac{25+n}{5n}}
%	\pairm{\dfrac{5}{n}-\dfrac{1}{5}}{\dfrac{25-n}{5n}}
%	\pairm{\dfrac{5}{n}\cdot\dfrac{1}{5}}{\dfrac{1}{n}}
%	\pairm{\dfrac{5}{n}:\dfrac{1}{5}}{\dfrac{25}{n}}	
%\pairm{x\cdot \dfrac{1}{x}}{1}
%\pairm{\dfrac{x}{2}}{\dfrac{3}{6}\cdot x}
%\pairm{\dfrac{\frac{1}{x}}{x}}{\dfrac{1}{x^2}}
%\pairm{\dfrac{1}{\frac{1}{x}}}{x}
%\pairm{\dfrac{4}{2x}:\dfrac{1}{x}}{2}
%\end{deck}

%murtolausekesievennyksiä 2
\begin{deck}{M}
\pairm{\dfrac{2x+6}{2}}{x+3}
\pairm{\dfrac{2x+6}{x+3}}{2}
\pairm{\dfrac{8x+12}{4}}{2x+3}
\pairm{\dfrac{8x+12}{2x+3}}{4}
\pairm{\dfrac{2x+6}{4}}{\dfrac{x+3}{2}}
\pairm{\dfrac{2}{y}:(2-y)}{\dfrac{2}{2y- y^2}}
\pairm{\dfrac{y+2}{y}}{1+\dfrac{2}{y}}
\pairm{\dfrac{y-y}{1+x}}{0}
\pairm{y^2:\dfrac{1}{y}}{y^3}
\pairm{1:y^2:\dfrac{1}{y}}{\frac{1}{y}}
\pairm{\dfrac{1}{y}:y^2}{\frac{1}{y^3}}
\pairm{z-\dfrac{z-1}{2}}{\dfrac{z+1}{2}}
\pairm{z-\dfrac{z-1}{3}}{\dfrac{2z+1}{3}}
\pairm{-\dfrac{z-2}{2}}{1-\dfrac{z}{2}}
\pairm{\dfrac{z(z-1)}{z}}{z-1}
\pairm{\dfrac{2z^2-2z}{z}}{2z-2}
\pairm{\dfrac{2z^2-2z}{2z-2}}{z}
\pairm{\dfrac{2z^2-2z}{2}}{z^2-z}
\end{deck}

\begin{deck}{N}
\pairm{y(y^2-1)}{y^3-y}
\pairm{y^2(y-1)}{y^3-y^2}
\pairm{y(y^2-1)}{(y^2+y)(y-1)}
\pairm{(y+1)^2}{y^2+1+2y}
\pairm{(y-1)^2}{y^2-2y+1}
\pairm{(y+1)^3}{y^3+3y^2+3y+1}
\pairm{(y+1)(y-1)}{y^2-1}
\pairm{y(1-y)}{-y^2+y}
\pairm{y^2-y}{y(y-1)}
\pairm{(1-y)^2}{y^2+1-2y}
\pairm{y-y(1-y)}{y^2}
\pairm{y^2-y^2(1-y^2)}{y^4}
\pairm{y^3\cdot y^2}{y^5}
\pairm{(-1-y)y}{-y-y^2}
\pairm{y-y^2}{(1-y)y}
\pairm{y-y(y+1)}{-y^2}
\pairm{y^2-y^2(y^2+1)}{-y^4}
\pairm{y^2-4}{(y+2)(y-2)}
\end{deck}

\begin{deck}{O}
\pair{yhteenlaskun liitännäisyys}{$ (a+s)+d=a+(s+d) $}
\pair{vähennyslasku}{$ \heartsuit-\spadesuit=\heartsuit+(-\spadesuit) $}
\pair{osittelulaki}{$ y(z+x)=yz+yx $}
\pair{laventaminen}{$ \dfrac{f}{g}=\dfrac{kf}{kg} $}
\pair{samankantaisten potenssien tulo}{$ t^xt^y=t^{x+y} $}
\pair{samankantaisten potenssien osamäärä}{$ \dfrac{w^a}{w^b}=w^{a-b} $}
\pair{tulon potenssi}{$ (st)^a=s^at^a $}
\pair{osamäärän potenssi}{$ \left (\dfrac{m}{n}\right )^q=\dfrac{m^q}{n^q} $}
\pair{potenssin potenssi}{$ (s^r)^t = s^{rt}$}
\end{deck}

\begin{deck}{P}%power rules 1, no fractional powers
\pairm{g^4g^4}{g^{2^3}}
\pairm{g^8g^2}{g^3g^7}
\pairm{{g^3}^2}{g^9}
\pairm{(g^3)^2}{g^6}
\pairm{\frac{g^{400}}{g^{250}}}{g^{150}}
\pairm{\frac{g^{250}}{g^{400}}}{g^{-150}}
\pairm{\frac{10^g}{10^{g-1}}}{10}
\pairm{\left( \frac{g}{2}\right )^{10}}{\frac{g^{10}}{1024}}
\pairm{(2g)^{10}}{1024g^{10}}
\end{deck}

\begin{deck}{Q}%power rules 2 – with roots and fractional powers
\pairm{\sqrt{\sqrt{x}}}{\sqrt[4]{x}}
\pairm{\sqrt{4^x}}{2^x}
\pairm{\sqrt{2^x}}{2^{\frac{1}{2}x}}
\pairm{\sqrt{9^x}}{3^x}
\pairm{\sqrt{(x^2)^3}}{x^3}
\pairm{\sqrt{\sqrt[3]{\sqrt[4]{x}}}}{x^\frac{1}{24}}
\pairm{\sqrt[x]{x^x}}{x}
\pairm{(-2)^\frac{2}{3}}{\text{ei hyvin määritelty}}
\pairm{x^{\frac{1}{2}+\frac{1}{3}}}{(x^\frac{5}{2})^\frac{1}{3}}
\end{deck}

%
%JOSTAIN SYYSTÄ \CE RIKKOO PAIR-KOMENNNON... MIKSI???

%\begin{deck}{Ke1} % molecular formula with structural formula 1
%%\pair{\chemfig{-}}{\ce{C2H6}}
%%\pair{\chemfig{=}}{\ce{C2H4}}
%%\pair{\chemfig{~}}{\ce{C2H2}}
%%\pair{\chemfig{-[:30]-[:-30]}}{\ce{C3H8}}
%%\pair{\chemfig{N~N}}{\ce{N2}}
%%\pair{\chemfig{-[:30]NH_2}}{\ce{CH5N}}
%%\pair{\chemfig{-[:30]OH}}{\ce{CH4O}}
%%\pair{\chemfig{-[:30]O-[:-30]}}{\ce{C2H6O}}
%%\pair{\chemfig{*4(----)}}{\ce{C4H8}}
%%\pair{\chemfig{*6(------)}}{\ce{C6H12}}
%%\pair{\chemfig{*6(=-=-=-)}}{\ce{C6H6}}
%%\pair{\small\chemfig{*5(---(-OH)--)}}{\ce{C5H10O}}
%%\pair{\chemfig{-[:60]=[0]-[:-60]}}{\ce{C4H8}}
%%\pair{\chemfig{-[:60]=[0](-[:60]F)-[:-60]}}{\ce{C4H7F}}
%%\pair{\small\chemfig{*6(---(<Cl)-(<:Cl)--)}}{\ce{C6H10Cl2}}
%%\pair{\small\chemfig{*6(---(<:Cl)-(<:Cl)--)}}{\ce{C6H10Cl2}}
%%\pair{\chemfig{-[:30](-[:60])(<:[:120]H)<[:-30]F}}{\ce{C3H7F}}
%%\pair{\chemfig{-[:30](-[:60]Cl)(<:[:120]H)<[:-30]F}}{\ce{C2H4ClF}}
%\end{deck}

%\begin{deck}{Ke2} %skeletal formula and condensed formula
%\pair{\chemfig{-}}{\ce{CH3CH3}}
%\pair{\chemfig{=}}{\ce{CH2CH2}}
%\pair{\chemfig{-[:30](-[2]OH)-[:-30]}}{\ce{CH3CH(OH)CH3}}
%\pair{\chemfig{-[:30]-[:-30]}}{\ce{CH3CH2CH3}}
%\pair{\chemfig{-[:30]-[:-30]-[:30]-[:-30]-[:30]-[:-30]}}{\ce{CH3(CH2)5CH3}}
%\pair{\chemfig{-[:30](-[2])-[:-30]-[:30]}}{\ce{(CH3)2CHCH2CH3}}
%\pair{\chemfig{-[:30]\chemabove{N}{H}-[:-30]-[:30]}}{\ce{CH3NHCH2CH3}}
%\pair{\chemfig{-[:30]OH}}{\ce{CH3OH}}
%\pair{\chemfig{-[:30]O-[:-30]}}{\ce{CH3OCH3}}
%\end{deck}

%\begin{deck}{Ke3} %structural/constitutional isomers
%\pair{\chemfig{*5(-----)}}{\chemfig{*4(---(-)-)}}
%\pair{\chemfig{*6(------)}}{\chemfig{*5(---(-)--)}}
%\pair{\chemfig{-[:30]-[:-30]-[:30]}}{\chemfig{-[:30](-[:120])-[:-30]}}
%\pair{\chemfig{-[:30]O-[:-30]}}{\chemfig{-[:30]-[:-30]OH}}
%\pair{\chemfig{-[:30]-[:-30]NH_2}}{\chemfig{-[:30]\chemabove{N}{H}-[:-30]}}
%\pair{\chemfig{-[:30](=[2]O)-[:-30]}}{\chemfig{-[:30]-[:-30]=[6]O}}
%\pair{\chemfig{*6(-=--=-)}}{\chemfig{*6(-=-=--)}}
%\pair{\chemfig{*4(--(=O)--)}}{\chemfig{*5(--O--=)}}
%\pair{\chemfig{-[:30](=[2]O)-[:-30]-[:30]OH}}{\chemfig{-[:30]-[:-30]([:30]-OH)=[6]O}}
%\end{deck}

%\begin{deck}{C4}%pair structural formulas with their enantiomers
%\pair{\chemfig{}}{\chemfig{}}
%\end{deck}
%
%\begin{deck}{C5}%pair structural formulas with their E-Z-isomers
%\pair{\chemfig{}}{\chemfig{}}
%\end{deck}
%
%\begin{deck}{C6}%pair structural formulas with their E-Z-isomers
%\pair{\chemfig{}}{\chemfig{}}
%\end{deck}


%\pairm{\dfrac{xxxx}{xx}}{xx}
%\pairm{\dfrac{xxx}{xxxx}}{\dfrac{1}{x}}



%\pairm{\dfrac{x\cdot x-1}{x-1}}{x+1}


%\begin{deck}{työläs}
%\pairm{\dfrac{1+x}{x}}{1+1/x}
%\pairm{1/(1+1/x)}{x/(x+1)}
%
%\pairm{\dfrac{3x^2}{x}}{3x}
%\pairm{\dfrac{3x^2}{6x}}{\dfrac{x}{2}}
%\pairm{\dfrac{3x}{x^2}}{\dfrac{3}{x}}
%\pairm{\dfrac{6x}{3x^2}}{\dfrac{2}{x}}
%
%\pairm{\dfrac{6x^2}{x}}{6x}
%
%\pairm{\dfrac{x-1}{x^2-1}}{\dfrac{1}{x+1}}
%\pairm{\dfrac{x-1}{x^2-x}}{\dfrac{1}{x}}
%\pairm{\dfrac{x-1}{x^2-x^2}}{\text{ei määritelty}}
%\pairm{\dfrac{\dfrac{x^3}{10}}{\dfrac{x}{5}}}{\dfrac{x^2}{2}}
%\pairm{\dfrac{\dfrac{1+x}{x}}{\dfrac{x^2+x}{x^2}}}{1}
%
%\pairm{\dfrac{x^2(x-1)}{x+1}\cdot \dfrac{x+1}{x^2}}{x-1}
%%\pairm{(x-a)(x-b)(x-c)\cdot \ldots \cdot (x-å)}{0}
%
%\pairm{\dfrac{1}{x}+\dfrac{1}{x^2}}{\dfrac{x+1}{x^2}}
%\pairm{\dfrac{1}{x}-\dfrac{1}{x^2}}{\dfrac{\dfrac{-2}{x^2}-\dfrac{-2}{x}}{2}}
%\pairm{\dfrac{1}{x}\cdot \dfrac{1}{x^2}}{\dfrac{1}{x^3}}
%\pairm{\dfrac{1}{x}:\dfrac{1}{x^2}}{x}
%
%\pairm{\dfrac{x^2-x+x(x+1)}{x^2}}{\dfrac{x-1+(x+1)}{x}}
%
%\pairm{\dfrac{x-1}{x}+\dfrac{x+1}{x}}{2}
%\pairm{\dfrac{x-1}{x}-\dfrac{x+1}{x}}{-\dfrac{2}{x}}
%\pairm{\dfrac{x-1}{x}\cdot \dfrac{x+1}{x}}{1-\dfrac{1}{x^2}}
%\pairm{\dfrac{x-1}{x}:\dfrac{x+1}{x}}{\dfrac{x-1}{x+1}}
%
%\pairm{\dfrac{x}{x-1}+\dfrac{x-1}{x}}{\dfrac{2x^2-2x+1}{x^2-x}}
%\pairm{\dfrac{x}{x-1}-\dfrac{x-1}{x}}{\dfrac{2x-1}{x^2-x}}
%\pairm{\dfrac{x}{x-1}\cdot \dfrac{x-1}{x}}{1}
%\pairm{\dfrac{x}{x-1}:\dfrac{x-1}{x}}{\dfrac{x^2}{x^2-2x+1}}
%
%\pairm{(a-1/a):(1-1/a)}{a+1}
%\end{deck}%54

%
%\begin{deck}{L}
%\pair{y=x^3-x}{
%\begin{tikzpicture}[smooth, scale=0.5] %domain=-1.75:1.75, 
%    \draw[very thin,color=gray] (-4,-4) grid (4,4);
%    \draw[->] (-4.2,0) -- (4.2,0) node[right] {$x$};
%    \draw[->] (0,-4.2) -- (0,4.2) node[above] {$y$};
%	\clip (-4,-4) rectangle (4,4);
%    \draw[color=red] plot[id=x] function{x**3-x} node[right] {};
%\end{tikzpicture}
%}
%
%\pair{y=1-x}{
%\begin{tikzpicture}[smooth, scale=0.5] %domain=-1.75:1.75, 
%    \draw[very thin,color=gray] (-4,-4) grid (4,4);
%    \draw[->] (-4.2,0) -- (4.2,0) node[right] {$x$};
%    \draw[->] (0,-4.2) -- (0,4.2) node[above] {$y$};
%	\clip (-4,-4) rectangle (4,4);
%    \draw[color=red] plot[id=x] function{1-x} node[right] {};
%\end{tikzpicture}
%}
%
%\pair{y=x}{
%\begin{tikzpicture}[smooth, scale=0.5] %domain=-1.75:1.75, 
%    \draw[very thin,color=gray] (-4,-4) grid (4,4);
%    \draw[->] (-4.2,0) -- (4.2,0) node[right] {$x$};
%    \draw[->] (0,-4.2) -- (0,4.2) node[above] {$y$};
%	\clip (-4,-4) rectangle (4,4);
%    \draw[color=red] plot[id=x] function{x} node[right] {};
%\end{tikzpicture}
%}
%
%\pair{y=2x}{
%\begin{tikzpicture}[smooth, scale=0.5] %domain=-1.75:1.75, 
%    \draw[very thin,color=gray] (-4,-4) grid (4,4);
%    \draw[->] (-4.2,0) -- (4.2,0) node[right] {$x$};
%    \draw[->] (0,-4.2) -- (0,4.2) node[above] {$y$};
%	\clip (-4,-4) rectangle (4,4);
%    \draw[color=red] plot[id=x] function{2*x} node[right] {};
%\end{tikzpicture}
%}
%
%\pair{y=x^2}{
%\begin{tikzpicture}[smooth, scale=0.5] 
%    \draw[very thin,color=gray] (-4,-4) grid (4,4);
%    \draw[->] (-4.2,0) -- (4.2,0) node[right] {$x$};
%    \draw[->] (0,-4.2) -- (0,4.2) node[above] {$y$};
%	\clip (-4,-4) rectangle (4,4);
%    \draw[color=red] plot[id=x] function{x**2} node[right] {};
%\end{tikzpicture}
%}
%
%\pair{y=x+1}{
%\begin{tikzpicture}[smooth, scale=0.5] %domain=-1.75:1.75, 
%    \draw[very thin,color=gray] (-4,-4) grid (4,4);
%    \draw[->] (-4.2,0) -- (4.2,0) node[right] {$x$};
%    \draw[->] (0,-4.2) -- (0,4.2) node[above] {$y$};
%	\clip (-4,-4) rectangle (4,4);
%    \draw[color=red] plot[id=x] function{x+1} node[right] {};
%\end{tikzpicture}
%}
%
%\pair{y=\lvert x \rvert}{
%\begin{tikzpicture}[ scale=0.5] %domain=-1.75:1.75, 
%    \draw[very thin,color=gray] (-4,-4) grid (4,4);
%    \draw[->] (-4.2,0) -- (4.2,0) node[right] {$x$};
%    \draw[->] (0,-4.2) -- (0,4.2) node[above] {$y$};
%	\clip (-4,-4) rectangle (4,4);
%    \draw[color=red] plot[id=x] function{abs(x)} node[right] {};
%\end{tikzpicture}
%}
%
%\pair{y=\lvert x +1\rvert}{
%\begin{tikzpicture}[scale=0.5] %domain=-1.75:1.75, 
%    \draw[very thin,color=gray] (-4,-4) grid (4,4);
%    \draw[->] (-4.2,0) -- (4.2,0) node[right] {$x$};
%    \draw[->] (0,-4.2) -- (0,4.2) node[above] {$y$};
%	\clip (-4,-4) rectangle (4,4);
%    \draw[color=red] plot[id=x] function{abs(x+1)} node[right] {};
%\end{tikzpicture}
%}
%
%
%\end{deck}



%\begin{deck}{R}
%
%\pair{\chemfig{H_2C=CH_2 }}{additio}
%
%\end{deck}





%\begin{deck}{D} %trigonometric identities
%\pair{\sin^2 x + \cos^2 x}{1}
%\pair{\cos^2 x \sin x + \sin^3 x}{\sin x}
%\end{deck}

%\begin{deck}{F1} %yhdistä kaavan nimi tai merkintä kaavaan
%\pair{paino}{$\vv{G}=m\vv{g}$}
%\pair{Newton II}{$\sum \vv{F}_i=m\vv{a}$}
%\pair{jousen potentiaalienergia}{$\Delta E_\text{p}=\dfrac{1}{2}k(\Delta x)^2$}
%\pair{harmoninen voima}{$\vv{F}=-k\Delta \vv{x}$}
%\pair{Newton III}{$ \vv{F}_{12}=-\vv{F}_{21} $}
%\pair{etenemisen liike-energia}{$E_\text{k}=\dfrac{1}{2}mv^2$}
%\pair{painovoiman potentiaalienergia\\ (approksimaatio)}{$ \Delta E_\text{p}=mg\Delta h $}
%\pair{mekaanisen energian säilymislaki}{$ \Delta E_\text{p}+\Delta E_\text{k}=0$}
%\pair{energian säilymislaki}{$ \sum E=\text{vakio (ajassa)} $}
%\pair{Newton I}{$ \sum \vv{F}=\vv{0} \Rightarrow (\vv{a}=\vv{0} \Longleftrightarrow \vv{v}=\text{vakio})$}
%\pair{Newtonin painovoimalaki}{$G=\gamma\dfrac{m_1m_2}{r^2}$}
%\pair{noste painovoimakentässä}{$ \vv{N}=\rho V \vv{g} $}
%\pair{normaali-ilmanpaine}{$ p_0 $}
%\pair{hydrostaattinen paine}{$p=\rho g h$}
%\pair{voiman tekemä työ}{$ W=\int\vv{F}\cdot \d\vv{s} \approx Fs$}
%\pair{teho}{$P=\dfrac{\d E}{\d t}$}
%\pair{liukukitka}{$ F_\upmu=\mu N $}
%\pair{paine}{$ P=\dfrac{F}{A} $}
%\pair{lepokitka}{$ F_{\upmu_0}=\mu_0 N $}
%\pair{keskinopeus}{$ \vv{v}_\text{k}=\dfrac{\Delta \vv{x}}{\Delta t} $}
%\pair{keskikiihtyvyys}{$ \vv{a}_\text{k}=\dfrac{\Delta \vv{v}}{\Delta t} $}
%\pair{(hetkellinen) nopeus}{$ \vv{v}=\dfrac{\dd \vv{x}}{\dd t} $}
%\pair{(hetkellinen) kiihtyvyys}{$ \vv{a}=\dfrac{\dd \vv{v}}{\dd t} $}
%\pair{liikemäärän säilymislaki}{$ \sum \vv{p}_\text{alussa} =\sum \vv{p}_\text{lopussa}$}
%\pair{impulssi}{$\vv{I}=\vv{F}\Delta t=\Delta \vv{p}$}
%\pair{Coulombin laki}{$ F=\dfrac{q_1q_2}{4\pi\epsilon_\text{r}\epsilon_0r^2} $}
%\pair{sijainti tasaisessa kiihdytyksessä}{$ \vv{x}=\vv{x}_0+\vv{v}_0 \Delta t+\dfrac{1}{2}\vv{a}(\Delta t)^2 $}
%\pair{nopeus tasaisessa kiihdytyksessä}{$ \vv{v}=\vv{v}_0+\vv{a}\Delta t $}
%\end{deck}



%todo: funktio- derivaattafunktioparit kuvaajien avulla

%todo (lopulta) korttipakan generointi niin, että on voinut valita flagit aksioomittain, merkinnöittäin, laskutoimituksittain jne.





\begin{deck}{R}%unit conversions – magnitudes and coefficients 1
\pairm{\mathrm{kg}}{1\,000\,\mathrm{g}}
\pairm{\mathrm{km}}{\text{miljoona}\,\mathrm{mm}}
\pairm{1\mathrm{dm^3}}{1\,\mathrm{l}}
\pairm{\mathrm{fl}}{10^{-15}\,\mathrm{l}}
\pairm{10\,\mathrm{dm^3}}{100\,\mathrm{dl}}
\pairm{\mathrm{\upmu m}}{10^{-6}\,\mathrm{m}}
\pairm{\mathrm{\upmu l}}{\mathrm{mm^3}}
\pairm{\mathrm{ha}}{10\,000\,\mathrm{m^2}}
\pairm{\mathrm{m^3}}{10^6\,\mathrm{ml}}
\end{deck}

%\begin{deck}{I}
%	\pair{Lausekkeen $ \frac{x+7}{x} $ osoittajan ja nimittäjän yhteinen tekijä?}{ei hyvin määritelty}
%	\pair{Lausekkeen $ xy+xz$ termien yhteinen tekijä?}{$ x $}
%\end{deck}


\begin{deck}{S}%linear equations 1
\pairm{x+1=2}{x=1}
\pairm{x+5=2}{x+3=0}
\pairm{x+y=2}{y=2-x}
\pairm{xy=3}{x=\frac{3}{y} \qq{($ y\neq0 $)}}
\pairm{y=\frac{2-x}{3}}{x=2-3y}
\pairm{xy+2y=5}{y(x+2)=5}
\pairm{(x+1)^2=2}{x^2+2x-1=0}
\pairm{2x-5=2}{x=\frac{7}{2}}
\pairm{1-x=99}{x=-98}
\end{deck}

\begin{deck}{T}%unit conversions – derivative units 1
\pairm{2\,\mathrm{\frac{m}{s}}}{7,2\,\mathrm{\frac{km}{h}}}
\pairm{1\,\mathrm{\frac{g}{ml}}}{1\,\mathrm{\frac{kg}{\ell}}}
\pairm{2\,\mathrm{\frac{g}{dm^3}}}{2\,\mathrm{\frac{kg}{m^3}}}
\pairm{3\,\mathrm{\frac{kpl}{nl}}}{3\cdot10^6\,\mathrm{\frac{kpl}{ml}}}
\pairm{10\,\mathrm{\frac{mol}{\ell}}}{10\,\mathrm{\frac{mmol}{ml}}}
\pairm{2\,\mathrm{\frac{\upmu g}{cm^3}}}{0,002\,\mathrm{\frac{g}{\ell}}}
\pairm{36\,\mathrm{\frac{ng}{nl}}}{36\,\mathrm{\frac{g}{dm^3}}}
\pairm{3000\,\mathrm{\frac{kg}{ha}}}{30\,\mathrm{\frac{kg}{a}}}
\pairm{1,2\,\mathrm{\frac{kg}{m^3}}}{1,2\,\mathrm{\frac{g}{\ell}}}
\end{deck}

%\begin{deck}{T} %electron configurations
%\pair{\ce{Cu}}{\elconf{Cu}}
%\pair{\ce{S}}{\elconf{S}}
%\pair{\ce{N}}{\elconf{N}}
%\pair{\ce{Ar}}{\elconf{Ar}}
%\pair{\ce{Ca}}{\elconf{Ca}}
%\pair{\ce{Li}}{\elconf{Li}}
%\pair{\ce{He}}{\elconf{He}}
%\pair{\ce{Cl}}{\elconf{Cl}}
%\pair{\ce{Ag}}{\elconf{Ag}}
%\end{deck}

%\begin{deck}{R} %form an equation based on text
%\pair{Lukujen $ a $ ja $ b $ tulo on $ 3 $.}{$ ab=3 $}
%
%\end{deck}


%\begin{deck}{P}
%\pairm{V(r)=\dfrac{4}{3}\pi r^3}{\mathbb{R}_+ \rightarrow \mathbb{R}_+}
%%\pairm{v(t)=\dfrac{\Delta x}{\Delta t}}{\mathbb{R}_+ \rightarrow \mathbb{R}}
%\pairm{v(\Delta x)=\dfrac{\Delta x}{\Delta t}}{\mathbb{R} \rightarrow \mathbb{R}}
%%\pairm{s(t)=vt}{\mathbb{R}_+ \rightarrow \mathbb{R}}
%%\pairm{s(v)=vt}{\mathbb{R} \rightarrow \mathbb{R}}
%\pairm{F(m)=ma}{\mathbb{R}_+ \rightarrow \mathbb{R}}
%\pairm{T(l)=2\pi \sqrt{\dfrac{l}{g}}  }{\mathbb{R}_+ \rightarrow \mathbb{R}_+}
%\pairm{\rho(m)=\dfrac{m}{V} }{\mathbb{R}_+ \rightarrow \mathbb{R}_+}
%\pairm{\rho(V)=\dfrac{m}{V}}{\mathbb{R}_+ \rightarrow \mathbb{R}_+}
%%\pairm{ N(V)=\rho V g}{\mathbb{R}_+ \rightarrow \mathbb{R}_+}
%\pairm{ E(v)=\dfrac{1}{2}mv^2}{\mathbb{R} \rightarrow \mathbb{R}\setminus \mathbb{R}_-}
%%\pairm{ P(t)=\dfrac{\Delta E}{t}}{\mathbb{R}_+ \rightarrow \mathbb{R}_+}
%\pairm{F(r)=\gamma \dfrac{m_1m_2}{r^2} }{\mathbb{R}_+ \rightarrow \mathbb{R}_+}
%\pairm{ \lambda(m)=\left( R_\mathrm{H} \left(\dfrac{1}{m^2}-\dfrac{1}{n^2}\right) \right)^{-1} $\vfill$ R_\mathrm{H}>0 $\vfill$ m,n\in \mathbb{N} }{\mathbb{N} \rightarrow \mathbb{R}_+}
%\pairm{E(m)=mc^2$ \vfill $\mathrm{c=valonnopeus}}{\mathbb{R}_+ \rightarrow \mathbb{R}_+}
%\pairm{T_{\frac{1}{2}}(\lambda)=\dfrac{\ln 2}{\lambda} $ \vfill $ \lambda>0}{\mathbb{R}_+ \rightarrow \mathbb{R}_+}
%\pairm{N(t)=N_0e^{-\lambda t}$ \vfill $\lambda, N_0>0}{\mathbb{R} \rightarrow \mathbb{R}_+}
%\pairm{f(b)=\left(\dfrac{1}{a}+\dfrac{1}{b}\right)^{-1} $\vfill$ a,b\in \mathbb{R}\setminus \lbrace 0 \rbrace}{\mathbb{R}\setminus \lbrace 0 \rbrace \rightarrow \mathbb{R} \setminus \lbrace 0 \rbrace}
%\end{deck}

%\begin{deck}{R}
%%\pairg{y=x^3-x}{x**3-2*x}
%%\pairg{y=x^2-3x+2}{x**2-3*x+2}
%%\pairg{y=(x-1)(x-2)(x-3)}{(x-1)*(x-2)*(x-3)}
%\pairg{y=x}{x}
%\pairg{y=-\frac{1}{2}x}{-0.5*x}
%\pairg{y=\sin(x)}{sin(x)}
%\pairg{y=\sin(x)+1}{sin(x)+1}
%\pairg{y=\sin(x+2)}{sin(x+2)}
%\pairg{y=\sin(2x)}{sin(2*x)}
%\pairg{y=2\sin(x)}{2*sin(x)}
%%\pairg{y=\cos(x)}{cos(x)}
%\pairg{y=\lvert x \rvert}{abs(x)}
%\pairg{y=\lvert x +1\rvert}{abs(x+1)}
%\end{deck}

%\begin{deck}{S}
%\pairm{$ y=\sin(x) $}{
%	\begin{tikzpicture}
%	\begin{axis}
%	\addplot{sin(x)}
%	\end{axis}
%	\end{tikzpicture}
%}
%\end{deck}

%function expression, domain/määrittelyehto
%function, graph

%structural formula with projection name
%structural formula with a some property (optically active, heterocyclic, hydrocarbon, ...)

%equations in power form, root form

%equations in power form, logarithmic form


%powers, roots and logarithms for purposes of using simple calculator and numeric tables



%\begin{deck}{I}
%	\pairm{\sqrt[3]{10}}{10^{0,333...}}
%	\pairm{\lg(0,0011)}{\lg(0,11)-2}
%	\pairm{\sqrt[5]{2}}{2^{0,2}}
%	\pairm{2^{0,3}}{10^{0,3\lg(2)}}
%	\pairm{\dfrac{10^7}{10^5\cdot 10^{-9}}}{10^{11}}
%	\pairm{10^{-0,4}}{\dfrac{1}{10^{0,4}}}
%	\pairm{\sqrt{10^{16}}}{10^8}
%	\pairm{\sqrt{3,2\cdot 10^{15}}}{\sqrt{32}\cdot10^7}
%	\pairm{10^{2,5}}{100\sqrt{10}}
%	\pairm{e^{\ln(10)}}{10}
%	\pairm{\ln(e^2)}{2}
%	\pairm{\sqrt[4]{10}}{\sqrt{\sqrt{10}}}
%	\pairm{\lg(11)}{\lg(0,11)+2}
%	\pairm{e^x}{10^{x\lg e}}
%	\pairm{e^{3,5}}{e^3\cdot \sqrt{e}}
%	\pairm{10^{-3,5}}{\dfrac{0,001}{\sqrt{10}}}
%	\pairm{e^{1,4}}{10^{\lg(e^{1,4})}}
%	\pairm{\ln(2)}{\dfrac{\lg(2)}{\lg(e)}}
%\end{deck}



%%%%%chemistry!!!
%\setbohr{insert-number=true, insert-symbol=true}
%\begin{deck}{C2}
%\pair{\bohr{1}{}}{\ce{H}}
%\pair{\bohr{2}{}}{\ce{He}}
%\pair{\bohr{3}{}}{\ce{Li}}
%\end{deck}



%\begin{deck}{K2} %names and formulas of ionic compounds
%	\pair{ammoniumnitraatti}{\ce{NH4NO3}}
%	\pair{kalsiumhydroksidi}{\ce{Ca(OH)2}}
%	\pair{litiumjodidi}{\ce{LiI}}
%	\pair{natriumvetykarbonaatti}{\ce{NaHCO3}}
%	\pair{kalsiumsulfaatti}{\ce{CaSO4}}
%	\pair{kalsiumsulfiitti}{\ce{CaSO3}}
%	\pair{kaliumnitriitti}{\ce{KNO2}}
%	\pair{kalsiumfosfaatti}{\ce{Ca3(PO4)2}}
%	\pair{kupari(II)sulfaattipentahydraatti}{\ce{CuSO4 * 5H2O}}
%\end{deck}

%\begin{deck}{K3} %names and formulas of (mainly) molecular compounds
%	\pair{trimetyyliamiini}{\ce{C3H9N}}
%	\pair{bentseeni}{\ce{C6H6}}
%	\pair{etikkahappo}{\ce{C2H4O2}}
%	\pair{natriumasetaatti}{\ce{CH3COONa}}
%	\pair{butanaali}{\ce{C4H8O}}
%	\pair{glyseroli}{\ce{C3H8O3}}
%	\pair{ammoniumsulfaatti}{\ce{(NH4)2SO4}}
%	\pair{sykloheksaani}{\ce{C6H12}}
%	\pair{2,2,3-trimetyyliheksaani}{\ce{C9H20}}
%\end{deck}

%\begin{deck}{K4} %reaction types: beginning substances and main product
%	\pair{karboksyylihappo + primäärinen alkoholi}{esteri}
%	\pair{alkoholi + alkoholi}{eetteri}
%	\pair{alkeeni + vety}{alkaani}
%	\pair{primäärinen amiini + karboksyylihappo}{amidi}
%	\pair{sekundäärinen alkoholi + hapetin}{ketoni}
%	\pair{alkyyni + pelkistin}{alkeeni}
%	\pair{alkeeni + vesi}{alkoholi}
%	\pair{aldehydi + hapetin}{karboksyylihappo}
%	\pair{primäärinen alkoholi + hapetin}{aldehydi}
%\end{deck}

%\begin{deck}{K5} %some chemical groups
%	\pair{butanaali}{aldehydi}
%	\pair{4-aminobutaanihappo}{karboksyylihappo, amiini}
%	\pair{glukoosi}{eetteri, alkoholi}
%	\pair{fenyylimetanoli}{aromaattinen, alkoholi}
%	\pair{fenoli}{aromaattinen, heikko happo}
%	\pair{heptaani}{alkaani}
%	\pair{\ce{CH3CH3SH}}{tioli}
%	\pair{etyylipropanoaatti}{esteri}
%	\pair{trikloorimetaani}{alkyylihalidi}
%\end{deck}