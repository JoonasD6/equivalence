\RequirePackage[l2tabu, orthodox]{nag}
\documentclass[10pt]{book}
\usepackage[utf8]{inputenc}
\usepackage{amsmath}
\usepackage{amsfonts}
\usepackage{amssymb}
\usepackage{makeidx}
\usepackage{fixmath}
\usepackage{multicol}
\usepackage{answers}
%\usepackage{luacode}
\usepackage{icomma}
\usepackage[pages=some]{background}
\usepackage{siunitx}
\usepackage{graphicx}
\usepackage{numprint}
\usepackage{tikz}
%\usepackage{tkz-euclide}
\usepackage[finnish]{babel}
 \usepackage{mathpazo}
\usepackage{fouriernc}
%\usepackage{xparse}
\usepackage{microtype}

\author{Joonas Mäkinen}
\title{Equivalent Expressions} %titlekortti

\Newassociation{vastaus}{Vastaus}{ans}

\newcounter{lauseke}
\setcounter{lauseke}{0}
\newcounter{kortteja} %mahdollisuus laittaa kolme varianttia lausekkeelle?
\setcounter{kortteja}{0}

\newcommand\lauseke[2]{

	\stepcounter{lauseke}
	\stepcounter{kortteja}

	\vspace*{\fill}
	\huge{\begin{center}$#2$\end{center}}
	\vspace*{\fill}

		\tiny{#1\arabic{lauseke}} \hfill	\logo \\
	\clearpage
}

\newcommand\lausekek[3][1]{

	\stepcounter{lauseke}
	\stepcounter{kortteja}

	\vspace*{\fill}
	\huge{\begin{center}
	\begin{tikzpicture}[smooth, scale=0.5]
    \draw[very thin,color=gray] (-#3,-#3) grid (#3,#3);
    \draw[->] (-4.2,0) -- (4.2,0) node[right] {\small{$x$}};
    \draw[->] (0,-4.2) -- (0,4.2) node[above] {\small{$y$}};
	\clip (-#3,-#3) rectangle (#3,#3);
    \draw[color=red] plot[id=x] function{#2} node[right] {};
	\end{tikzpicture}
	\end{center}}
	\vspace*{\fill}

		\tiny{#1\arabic{lauseke}} \hfill	\logo \\
	\clearpage
}

\newcommand\tausta{
		
		\BgThispage
		\tiny{
	\vspace*{\fill}
	 \hfill Joonas ''JoonasD6'' Mäkinen \\

	 \hfill Creative Commons-BY 4.0\\}
	\clearpage
}

\newcommand\kortti[2]{

\stepcounter{luokka#1}
\lauseke{#1}{#2}
%\tausta	
}

\newcommand\korttik[3]{

\stepcounter{luokka#1}
\lausekek{#1}{#2}{#3}
%\tausta	
}


%minne \begin{vastaus} \end{vastaus}
	
\newenvironment{kortit}[1]{

\newcounter{luokka#1}
\setcounter{luokka#1}{0}
		
		\renewcommand\k[2]{
			\kortti{#1}{##1}
			\kortti{#1}{##2} %entä kaksipuolisten korttien versio?
			}
		\renewcommand\kk[3][1]{
			\kortti{#1}{##1}
			\korttik{#1}{##2}{##3} %entä kaksipuolisten korttien versio?
			}
		}{
\setcounter{lauseke}{0}
}